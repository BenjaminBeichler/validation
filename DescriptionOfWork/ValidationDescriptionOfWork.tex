\documentclass{template/openetcs_article}
% Use the option "nocc" if the document is not licensed under Creative Commons
%\documentclass[nocc]{template/openetcs_article} 
\usepackage{lipsum,url}
\usepackage{xspace}
\usepackage{graphicx}
\usepackage{fixme}
\usepackage{lscape} 
\usepackage{pgfgantt}
\usepackage{adjustbox}
\usepackage{datetime}



%user specified macros
\newenvironment{activity}[2][planned]
	{\begin{tabular}{p{0.25\textwidth}@{\hspace{0.05\textwidth}}p{0.7\textwidth}}
			\multicolumn{2}{p{\textwidth}}{\colorbox{black}{\begin{minipage}{1.1cm}\begin{center}\textsc{\footnotesize \textcolor{white}{#1}}\end{center}\end{minipage}}~~\textbf{#2}}\\
	}
	{\end{tabular}}

\newcommand{\entry}[2]{#1:&#2\\}
\newcommand{\website}[1]{Website:&\url{#1}\\}
\newcommand{\desc}[1]{\multicolumn{2}{p{\textwidth}}{#1}\\}

\newcommand{\VV}{Verification \& Validation\xspace}
\newcommand{\vv}{verification \& validation\xspace}

\newcommand{\tbd}{\colorbox{cyan}{\%\%To Be Defined\%\%}}
\newcommand{\tbc}{\colorbox{cyan}{\%\%To Be Confirmed\%\%}}
\newcommand{\todo}[1]{\colorbox{cyan}{\%\%{#1}\%\%}}
\newcommand{\nthng}[1]{}


\graphicspath{{./template/}{.}{./images/}}
\begin{document}
\frontmatter
\project{openETCS}

%Please do not change anything above this line
%============================
% The document metadata is defined below

%assign a report number here
\reportnum{OETCS/WP4/DescriptionOfWork}

%define your workpackage here
\wp{Work Package 4: ``Validation \& Verification Strategy''}

%set a title here
\title{openETCS Validation \& Verification Strategy Work Package}

%set a subtitle here
\subtitle{Description of Work}

%set the date of the report here
\date{March 2012}

%define a list of authors and their affiliation here

\author{Marc Behrens}

\affiliation{WP4 Leader}
 
\author{Hardi Hungar}

\affiliation{WP4.1 Task Leader (Idetntification of tools and profile usage)}

\author{\ }

\affiliation{WP4.2 Task Leader (\VV of the formal model )}
 
\author{Jens Gerlach}

\affiliation{WP4.3 Task Leader (\VV of the implementation \/ code)}

\author{Hansj\"{o}rg Manz, Jan Welte}

\affiliation{WP4.4 Task Leader (Verification of the tools and processes)}

\author{Cyril Cornu }

\affiliation{WP4.5 Task Leader (Internal Assessment)}
  
% define the coverart
\coverart[width=350pt]{chart}

%define the type of report
\reporttype{Description of work}



\begin{abstract}
%define an abstract here
This work package will focus on the validation and verification of the model. At the very first beginning the target and requirements of the \VV strategy have to be described, i.e., what should be checked? Depending on the Modelling framework, the modelling language and formalization of the System requirements a strategy in form of a concept has to be defined how the consistency, coherence of the model as well as the coverage of system requirements will be transparently verified. Additionally, it is important to validate the model, i.e., to evidence the equivalence of the model and the ETCS system requirement specification (Subset-026 et al.). In other words the reliability and acceptance of the model has to be generated, e.g. nothing is lost or added or mutated and so on. Additionally, it has to be checked that the code is consistent with the model. The WP is intended to be performed in parallel with the modelling in order to apply the strategy and to generate feedback to the modelling process as well as to measure the quality and maturity of the model.
Beside a subtask will manage the consideration of all relevant safety requirements (e.g. EN 50128/129) in the modelling process.
\end{abstract}

%=============================
%Do not change the next three lines
\maketitle
\tableofcontents
\listoffiguresandtables
\newpage
%=============================

% The actual document starts below this line
%=============================


%Start here

%-----------------------------------------------------------------------
\section*{Introduction}
%-----------------------------------------------------------------------

\subsection*{Objectives}

\emph{Verification} is the activity to ascertain that a particular
step in the development has achieved its goals, i.e., that its result
correctly refines or implements its input, which may be a higher-level
design or a specification. \emph{Validation} is about making sure that
the end result of the development meets its initial specification,
that is, the requirements of the user. The term validation is also
used when a design artifact is checked against requirements from
previous steps. Verification and/or validation is required for most
development artifacts.  What exactly has to be checked depends on the
set of items produced in the development process, their role and their
nature.  As the EVC software contributes to several safety-critical
functions of the ETCS onboard unit, the specific requirements
concerning the safety aspects of the standards EN~50128 and EN~50129
have to be respected throughout.

A main obligation of this work package is the verification or
validation of development artifacts produced by WP~3. This work will
concentrate on the functional and safety aspect. Besides \vv,
the work package shall also establish a coherent and comprehensive
chain of methods and tools for V\&V in cooperation with WP~7. Specific
challenges in this respect arise from the wish to use models
extensively in the development process, which means that more common
approaches have to be improved or substituted to fit a
model-based development style, and from the requirement of
using open-source tools, or even trying to realize the EVC software as
an \emph{open proof} item.

In pursuing these goals, the work package generates feedback
concerning the adequacy, correctness and safety of development
artifacts for WP~3, and the usefulness of tools and methods for WP~7.  

\subsection*{Organisation of the Work package}

The work packages consists of five tasks. The first task defines the
\vv strategy and formulates the initial \vv plan. This plan defines the
\vv activities to be done in openETCS and proposes the means to
perform them. In later stages of the project the plan will be extended
and revised to reflect the findings made while applying methods and
tools to the artifacts at hand.

Both model and code of the EVC software are subjected to \vv as they
are produced by WP~3, and the tools and methods proposed in the \vv
plan as well as newly developed or improved tools from WP~7 are applied
and evaluated in the process. Findings from these steps are
iteratively fed back to the respective work package and used to refine
the \vv plan.

A dedicated activity studies the safety aspect of \vv. It takes into
cosideration what the standards (mainly EN~50128 and EN~50129) mandate
and defines how these requirements can be met by the combination of
life cycle, methods and tools.

An internal assessment will simulate a real Assessor's task doing a 
Software Development assessment of the project impacting Working Packages 
1, 2, 3, 4, 5, 6 and 7.

The phases defining the \vv process is divided into the design phase and 
the application phase. The \emph{design phase} covers the time before the 
release of the artifacts which are to be evaluated. In this phase the 
findings of the last application phase are taken into account to improve 
\vv.
The \emph{application phase} covers the time after the release of the 
artifacts to be evaluated until the \vv report is written. For this phase 
the artifacts to be evaluated are frozen to a fixed release date.  

%-----------------------------------------------------------------------
\subsection*{Techniques for \VV}
%-----------------------------------------------------------------------

\VV techniques can be roughly classified into \emph{dynamic} and
\emph{static} techniques.  The most common dynamic \vv techniques are
various forms of \emph{testing}, which execute the code or the
model. They are classfied by their object or their purpose. These
include:
\begin{itemize}
\item Unit testing
\item Integration testing
\item Acceptance test
\item Software-in-the-Loop
\item Model-in-the-Loop
\item Model-based testing
\item Monitoring
\item Coverage analysis
\end{itemize}
A related dynamic activity is \emph{animation}, which may play a role
in analysing an executable model.

Static \vv techniques---not executing model or code---include:
\begin{itemize}
\item Checking of coding guidelines
\item Review
\item Walkthrough
\item Formal methods
\begin{itemize}
\item Model checking
\item Deductive verification (theorem proving)
\item Abstract interpretation
\end{itemize}
\end{itemize}

%\begin{figure}[h]
%\centering
%\input{sections/openETCSOpenProofsDevelopmentProcess.pdf_tex}
%\caption{openETCS open proofs concept}
%\end{figure}

\todo{description on V\&V classification non formal-> formal -> formal
  -> code \& description}

\subsection*{Coping with a Model-Based Development Style}

Models appear at different stages of the development. An important
artifact of openETCS is a semi-formal model of the requirements. 
Depending on the modelling framework, the modelling language and
formalization of the system requirements a
concept has to be defined how the consistency and coherence of the
model as well as the coverage of system requirements will be
transparently verified. For this task, static verification techniques
will very likely offer the best approach.

To verify that the model correctly captures the ETCS system
requirement specification (Subset-026 et al.), also dynamic techniques
like animation might be useful. And finally, it may be helpful to also
validate the model against the user requirements.

For later development stages, correct refinement or implementation of
the model will have to be established. Again, techniques to be applied
depend heavily on the nature of the model(s) and the process of how
the code is derived. Model-based testing, i.e., deriving test cases
from a model to ascertain that an executable behaves consistent to a
model, is a technique to be used. Alternatively, if code is generated
automatically from a model, other means like tools checking the
correctness of a generation procedure (or its outcome on a
case-by-case basis) may be chosen.

An important issue to be kept in mind is the suitability of models
and tools for a safety-critical development. Modelling languages that
lack a formal semantics or the expressive power to capture system
aspects essential for safety considerations are of limited
usefulness. And tools need to be qualifiable according to their role.  
For instance, a code generator needs to be verified or qualified or it
must be accompanied by some tool checking the correctness of the
generation step. Otherwise, the resulting code will have to verified
similar as manually written code.





%-----------------------------------------------------------------------
\section{Identification of Tools and Profile Usage}
%-----------------------------------------------------------------------
The objective of this task is to prepare the activities of the tasks
4.2 and 4.3, which are concerned with \vv of the model and the code,
respectively. It defines an overall \vv strategy and plans for
verification and validation, detailing how and with what means the
strategy is going to be implemented. Formally, the requirements on \vv
which are to be covered in the plans are listed in D2.9 ``Requirements
for \VV''. An essential requirement is adhering to
the applicable standards, mainly the EN~50128. The plans shall define
activities adequate for a complete development, but also foresee a
tailoring to the partial development actually realised in the project.

The WP~2 deliverable D2.3 ``Process Definition'' defines the
development process and its steps, and thus also identifies the main
verification steps. These will be detailed in the verification plan,
defining what has to be achieved on a more technical level. Also, a
selection of potentially applicable tools and methods will be
given. The validation plan shall take the overall development approach
including verification activities into account and define what methods
are suitable for demonstrating that all requirements (to be defined by
WP~2) are met by the end result, and also address the question of
safety integrity.

Thus, the plans shall contain a selection of methods and a list tools
suitable for applying the chosen V\&V methods for\footnote{Terms
  according to the draft of D2.3, abbreviations according to the draft
  of D2.6.}
\begin{enumerate}
\item the sub-system requirements specification (SSRS) and models
  (SFM, FFM)
\item the software semi-formal model and software architecture
  description
\item code derived from the software semi-formal model
\item the software strictly formal model and the software design
  description
\end{enumerate}

D2.1 ``Report on Existing Methodologies'' shall already provide a list
of potentially relevant methods and tools. Each method and tool
applied in WP~4 shall be described in a format detailing its purpose,
role and characteristics in terms of requirements on \vv.  Not all
steps will be automatic or semi-automatic. Manual techniques
like review or walkthrough will play a role, too. In selecting tools,
besides the requirement of openness (FLOSS), the question of
qualification, depending on the role the tool will play, has to be
answered. The formats describing methods and tools and criteria for
their evaluation will be given in a document D4.1a ``Preliminary
Evaluation Criteria on Verification and Validation'' supplementing the
deliverable to be produced by this task, D4.1 ``Report on V\&V Plan \&
Methodology''. 

Important classes of objects subjected to \vv activities are the
different models, be they semi-formal or formal, the code derived from
them and the versions of the demonstrator. Similar to methods and
tools, these objects need to be defined w.r.t.\ their nature and role
in the development process, including the requirements for \vv. A
format for their description shall be provided with D4.1.

An analysis of these objects and the methods with which they are
developed shall lead to a refinement and concretisation of the
verification and validation plans in the course of the project after
termination of T4.1.


\begin{table}[h]
\caption{T4.1 Inputs, Outputs and Deliverables} %title of the table
\begin{adjustbox}{width=\textwidth}
\begin{tabular}{|l|l|r|r|r|}
\hline
\multicolumn{5}{|c|}{\textbf{T4.1 Identification of Tools and Profile Usage}} 
\\\hline
Type & Description & Due Date & Due Month & status 
%status output going to other tasks/wps    : not started, started, complete
%status input coming from other tasks/wps: no, yes
%\\\hline
%$\rightarrow$ & \todo{Ox.2.3: Sample Input Information}  & \shortmonthname[1]-2014 & T0+19 & no 
%\\\hline
%$\leftarrow$ & \todo{O4.1.1: Sample Output Information}   & \shortmonthname[10]-2013  & T0+16 & started  
\\\hline
$\rightarrow$ & \emph{D 2.1} Report on Existing Methodologies & \shortmonthname[3]-2013 & T0+9 &  no
\\\hline
$\rightarrow$ & \emph{D 2.3} Process Definition & \shortmonthname[5]-2013 & T0+11 &  no
\\\hline
$\rightarrow$ & \emph{D 2.4} Methods Definition & \shortmonthname[5]-2013 & T0+11 &  no
\\\hline
$\rightarrow$ & \emph{D 2.9} Set of Requirements for V\&V & \shortmonthname[5]-2013 & T0+11 &  no
\\\hline
 D &\emph{D 4.1a} Preliminary Evaluation Criteria on V\&V & \shortmonthname[3]-2013 & T0+9 & started
\\\hline
  D &\emph{D 4.1} Report on V\&V Plan \& Methodology  & \shortmonthname[7]-2013 & T0+13 & started
\\\hline
\end{tabular}
\end{adjustbox}
\end{table}

%-----------------------------------------------------------------------
\section{\VV of the Formal Model}
%-----------------------------------------------------------------------
\tbc
To ensure the correctness and consistency of the model and its implementation, the validation and verification has to be performed alongside with the modelling process. Thus these tasks will be performed repeatedly during WP3 and will provide feedback to it.

This task handles the verification and validation of the formal model. This will be accomplished by applying the methods chosen in WP4 Task 1 onto the formal model from WP3 using the tool chain developed in WP3. Depending on the chosen approach and applicable tools a variety of verification methods can be applied like:
\begin{enumerate}
\item proof technique
\item model checking technique
\item Simulation
\end{enumerate}
As the verification and validation is part of the development chain, this task is being applied iteratively in parallel to the development of the formal model in WP3. The feedback given should focus on the consistency and correctness of the model and development process in WP3.
The results of this task are the verification and validation specifications (how to perform the V\&V on the formal model), the basic materials (the actual tests cases, checklists, etc.) and the V\&V report on the formal model.

%-----------------------------------------------------------------------
%\subsection{\tbd}
%-----------------------------------------------------------------------
%\tbd
%Should Risk Assessment be done in parallel to the verification activities?



\begin{table}[h]
\caption{T4.2 Inputs, Outputs and Deliverables} %title of the table
\begin{adjustbox}{width=\textwidth}
\begin{tabular}{|l|l|r|r|r|}
\hline
\multicolumn{5}{|c|}{\textbf{T4.2 \VV of the Formal Model}} 
\\\hline
Type & Description & Due Date & Due Month & status 
%status output going to other tasks/wps    : not started, started, complete
%status input coming from other tasks/wps: no, yes
%\\\hline
%$\rightarrow$ & \todo{Ox.2.3: Sample Input Information}  & \shortmonthname[1]-2014 & T0+19 & no 
%\\\hline
%$\leftarrow$ & \todo{O4.2.1: Sample Output Information}   & \shortmonthname[10]-2013  & T0+16 & started  
\\\hline
D & \emph{D 4.4} Final report on \VV of the model  & \shortmonthname[7]-2013 & T0+13 & not started
\\\hline
\end{tabular}
\end{adjustbox}
\end{table}

%-----------------------------------------------------------------------
\section{V\&V of the Implementation \& Code}
%-----------------------------------------------------------------------

The objective of this task is to verify and validate the actual implementation of the formal model.
Therefore the tool chain from WP3\slash WP7 will be used to apply the chosen methods from
WP4 Taskr~ 1 onto the implementation of the formal model from WP3.
The chosen combination of methods and tools in WP4 Task 1 can result
in a wide variety of techniques to be used:

\begin{itemize}
\item Software-in-the-Loop
\item  Model-in-the-Loop
\item  Model-based testing
\item  Static analysis (e.g. coding guidelines)
\item  Formal methods (abstract intertpretation, deductive verification, model checking)
\item  Monitoring
\end{itemize}

Analogue to WP4 Task 2 the verification and validation of the formal model
implementation is part of the development chain.
Therefore this task runs parallel to the development of the formal model in WP3, and is being applied iteratively.
Therefore feedback regarding the validity and correctness is reported to the
development process in WP3.
The results of this task are the verification and validation specifications, that is,

\begin{itemize}
\item 
how to perform the V\&V on the formal model implementation,

\item
the basic materials (the actual tests cases, checklists, etc.) and

\item
the V\&V report on the implementation of the formal model.
\end{itemize}

As first steps the relevant properties and techniques concerning the code and
implementation are to be identified. 
We list therefore the following software properties that we think are most relevant for \vv:

\begin{itemize}
\item functionality
\item robustness (absence of runtime errors)
\item performance
\item real time behaviour
\item dataflow
\item absence of deadlocks
\end{itemize}

Table~\ref{tbl:task43} shows the main deliverables of Task~4.3.

\begin{table}[h]
\begin{adjustbox}{width=\textwidth}
\begin{tabular}{|l|l|r|r|r|}
\hline
\multicolumn{5}{|c|}{\textbf{T4.3 V\&V of the implementation \/ code}} 
\\\hline
Type & Description & Due Date & Due Month & status 
\\\hline
%status output going to other tasks/wps    : not started, started, complete
%status input coming from other tasks/wps: no, yes
 D &\emph{D 4.2} Interim report on the applicability of the V\&V approach  & \shortmonthname[12]-2013 & T0+18 & started
\\\hline
 D &\emph{D 4.4} Final Report on the V\&V of the Implementation & \shortmonthname[1]-2015 & T0+31 & not started
\\\hline
\end{tabular}
\end{adjustbox}
\caption{\label{tbl:task43} Inputs, Outputs and Deliverables of Task 4.3}
\end{table}


%-----------------------------------------------------------------------
\section{Verification of the Tools and Processes}
%-----------------------------------------------------------------------
\tbc
The software will be developed according to the guidelines specified in the CENELEC Standard 50128. Each of the Lifecycle stages (SW Requirement Specification, SW Design, SW Coding, etc.) must be fully documented and simultaneously verification and validation tasks must be performed.
In this task the safety management team draws a safety plan to identify the safety management structure, safety related activities and safety approval milestones. A hazard log will be created and maintained throughout the whole development process. In addition, the safety plan will include plans for verifying that each development phase meets its safety requirements. The safety plan also describes (among others):
\begin{enumerate}
\item  Roles, responsibilities and competences of the involved bodies
\item  Safety-related deliverables with milestones
\item  Procedures of preparing the safety case
\item  Procedures for maintaining safety documents
\end{enumerate}
All safety principles followed in the development process will be described along with documented quantitative analyses. Evidences of technical safety shall describe the safeguards used for individual safety properties. The V\&V reports are to be referred in this part.
Concerning the applied tools, 3rd parties may be engaged to perform the V\&V. Certainly, the respective results will be referred to in the safety case.


%Specify the tool classes and the potentials to Categorize which tools have to be T3 and which T2.}

%Safety Evaluation criteria

\begin{table}[h]
\caption{T4.4 Inputs, Outputs and Deliverables} %title of the table
\begin{adjustbox}{width=\textwidth}
\begin{tabular}{|l|l|r|r|r|}
\hline
\multicolumn{5}{|c|}{\textbf{T4.4 Verification of the Tools and Processes}} 
\\\hline
Type & Description & Due Date & Due Month & status 
%status output going to other tasks/wps    : not started, started, complete
%status input coming from other tasks/wps: no, yes
%\\\hline
%$\rightarrow$ & \todo{Ox.2.3: Sample Input Information}  & \shortmonthname[1]-2014 & T0+19 & no 
%\\\hline
%$\leftarrow$ & \todo{O4.2.1: Sample Output Information}   & \shortmonthname[10]-2013  & T0+16 & started  
\\\hline
D & \emph{D 4.4} Final report concerning the Safety Case  & \shortmonthname[6]-2015 & T0+36 & \tbd
\\\hline
\end{tabular}
\end{adjustbox}
\end{table}

%-----------------------------------------------------------------------
\section{Internal Assessment}
%-----------------------------------------------------------------------
\todo{Assessment abstract}

%-----------------------------------------------------------------------
\subsection{Assessment \todo{tasks}}
%-----------------------------------------------------------------------

Internal Assessment description
\tbd


\begin{table}[h]
\caption{T4.5 Inputs, Outputs and Deliverables} %title of the table
\begin{adjustbox}{width=\textwidth}
\begin{tabular}{|l|l|r|r|r|}
\hline
\multicolumn{5}{|c|}{\textbf{T4.5 Internal Assessment}} 
\\\hline
Type & Description & Due Date & Due Month & status 
%status output going to other tasks/wps    : not started, started, complete
%status input coming from other tasks/wps: no, yes
%\\\hline
%$\rightarrow$ & \todo{Ox.2.3: Sample Input Information}  & \shortmonthname[1]-2014 & T0+19 & no 
%\\\hline
%$\leftarrow$ & \todo{O4.5.1: Sample Output Information}   & \shortmonthname[10]-2013  & T0+16 & started  
\\\hline
 D &\emph{D 4.5} Quality recommendation to prepare the Assessment  & \tbd & \tbd & 
\\\hline
\end{tabular}
\end{adjustbox}
\end{table}






\noindent{
\begin{landscape}
%-----------------------------------------------------------------------
\section{GANTT chart}
%-----------------------------------------------------------------------

\begin{table}[h]
%\caption{WP4 GANTT chart} %title of the table
\begin{adjustbox}{height=\textheight/4*3}% ajusting graphic size for landscape
%\begin{adjustbox}{width=\textwidth}% ajusting graphic size for non-landscape
\begin{tikzpicture}[x=.5cm, y=1cm]
\begin{ganttchart}%
[hgrid=true, vgrid={*5{dotted},*1{solid},*5{dotted},*1{dashed}},%
today=8,
today label=\textcolor{blue}{Current Month}, 
today rule/.style={blue, line width=3pt},
y unit title=0.4cm,
y unit chart=0.5cm,
title label anchor/.style={below=-1.5ex},
title height=1,
bar height=.6,
bar label font=\normalsize\color{black!80},
milestone height=.6,
milestone yshift=.6,
milestone/.style={fill=black,draw=black},
group right shift=0,
group top shift=.6,
group height=.3,
group peaks={}{}{.2},
inline
]{36} %36 months

% project title
\gantttitle[title/.style={draw=none}, title height=1,
title label font={\color{black}\scshape}%
]{Verification \& Validation Strategy}{36} \\

%timing header
\gantttitle{2012}{6}
\gantttitle{2013}{12}
\gantttitle{2014}{12}
\gantttitle{2015}{6} \\
\gantttitlelist[title height=1]{7,...,12}{1}
\gantttitlelist[title height=1]{1,...,12}{1}
\gantttitlelist[title height=1]{1,...,12}{1}
\gantttitlelist[title height=1]{1,...,6}{1} \\
\gantttitlelist[title height=1]{1,...,36}{1} \\

%project groups and tasks
\\
\ganttbar[name=T41, inline=false]{Idetntification of tools and profile usage \emph{T4.1}}{7}{13} 
\ganttmilestone[name=D41a, bar label inline anchor/.style=left]{\emph{D 4.1a}}{9}  %Preliminary Evaluation criteria on V\&V \emph{D 4.1a}
\ganttmilestone[name=D41,  bar label inline anchor/.style=left]{\emph{D 4.1}}{13} \\ %V\&V Plan \& Methodology \emph{D 4.1}
\\
\ganttgroup[]{V\&V of prototypical Model}{14}{16} \\
\\
\ganttgroup[]{V\&V of Model \& Functional API propotype}{20}{24} \\
\\
\ganttgroup[]{V\&V of Model \& Functional API final}{32}{35} \\
\\
\ganttbar[name=T42, inline=false]{V\&V of the formal model \emph{T4.2 }}{10}{35}
\ganttmilestone[name=D42]{\emph{D 4.2}}{16} %Interim report on the applicability of the V\&V approach \emph{D 4.2}
\ganttmilestone[name=M41]{\emph{M 4.1}}{24} %Applicability of the V\&V approach to the prototype \emph{M 4.1}
\ganttmilestone[name=D43, , milestone label inline anchor/.style={anchor=south east}]{\emph{D 4.3}}{35} %Report on the prototypical application of the V\&V \emph{D 4.3}
\ganttmilestone[name=D45]{{\emph{D 4.5}}}{36} \\ %Final report and conclusions \emph{D 4.5}
\\
\ganttgroup[]{V\&V of prototypical Code \& API}{14}{16} \\
\\
\ganttgroup[]{V\&V of Architecture \& System API propotype}{21}{24} \\
\\
\ganttgroup[]{V\&V of Architecture \& System API final}{32}{35} \\
\\
\ganttbar[name=T43, inline=false]{V\&V of the implementation \/ code \emph{T4.3}}{10}{35} 
\ganttmilestone[name=D42]{\emph{D 4.2}}{16} %Interim report on the applicability of the V\&V approach \emph{D 4.2}
\ganttmilestone[name=M41]{\emph{M 4.1}}{24} %Applicability of the V\&V approach to the prototype \emph{M 4.1}
\ganttmilestone[name=D43, milestone label inline anchor/.style={anchor=south east}]{\emph{D 4.3}}{35} %Report on the prototypical application of the V\&V \emph{D 4.3}
\ganttmilestone[name=D45]{{\emph{D 4.5}}}{36} \\ %Final report and conclusions \emph{D 4.5}
\\
\ganttbar[name=T44, inline=false]{Verification of the tools and processes \emph{T4.4}}{7}{35} 
\ganttmilestone[name=D41a]{\emph{D 4.4a}}{9} %Preliminary Evaluation criteria on safety \emph{D 4.4a}
\ganttmilestone[name=D43, milestone label inline anchor/.style={anchor=south east}]{\emph{D 4.3}}{35} %Report on the \ganttmilestone[name=D45]{{\emph{D 4.5}}}{36} \\ %Final report and conclusions \emph{D 4.5}
\\
\ganttgroup[]{\tbd}{8}{35} \\
\\
\ganttbar[name=T45, inline=false]{Internal Assessment \emph{T4.5}}{8}{35}
\ganttmilestone[name=D45]{\emph{\todo{D 4.5}}}{35} \\  %Quality recommendation to prepare the Assessment \emph{D 4.5}

%project linking between tasks
%\ganttlink{D41a}{T43}
%\ganttlink[link type=s-s]{T41}{T42}

\end{ganttchart}
\end{tikzpicture}
\end{adjustbox}
\end{table}

\end{landscape}
}

\nocite{*}
%===================================================
%Do NOT change anything below this line

\end{document}
