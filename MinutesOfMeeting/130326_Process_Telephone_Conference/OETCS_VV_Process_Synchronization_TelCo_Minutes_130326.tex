\documentclass[a4paper]{article}
%\documentclass[a4paper,german]{article}
\usepackage{graphicx}
\usepackage{xspace}
\usepackage{longtable}
\usepackage{array}
\usepackage{xifthen}% provides \isempty test


\pagestyle{myheadings}
\markright{openETCS \hfill {\tiny This work is licensed under a Creative Commons Attribution-ShareAlike 3.0 Unported License} \hfill}


%\usepackage{amsmath}
%\usepackage{amssymb}
% % Deutsche Silbentrennung
% \usepackage[ngerman]{babel}
% % Deutsche Umlaute
% \usepackage[ansinew]{inputenc}
% %\usepackage[latin1]{inputenc}


\setlength{\parindent}{0pt}
\setlength{\parskip}{3pt}

% editing

% Starts a new line nearly everywhere
\newcommand{\nl}{\mbox{}\\}

%Texts in a box (eg. for comments)
% Short text (no line break) 
\newcommand{\cmmnt}[1]{\framebox{#1}}
% Long text (separate lines
\newcommand{\bgcmmnt}[1]{\nl\framebox{\parbox{.95\textwidth}{#1}}\nl[2mm]}

%Uncomment for getting rid of comments in output
%\renewcommand{cmmnt}[1]{}
%\renewcommand{\bgcmmnt}[1]{}

% Macros for minutes
\newcommand{\Q}[2]{\paragraph{Question} 
	\ifthenelse{\isempty{#1}}%
    	{}% if #1 is empty
    	{by #1}% if #1 is not empty
    : #2}
\newcommand{\A}[2]{\newline{\textbf{Answer}}
	\ifthenelse{\isempty{#1}}%
    	{}% if #1 is empty
    	{by #1}% if #1 is not empty    
    : #2}
\newcommand{\C}[2]{\textbf{Comment} by #1: #2}

% End of document marker
\newcommand{\eod}{\rule{\textwidth}{1pt}\nl \textit{End of Document}}

\begin{document}
\title{Telephone conferenc eon V\&V Process Synchronization  \\openETCS TelCo}
\author{Marc Behrens}
\date{Version 01, 2013-03-26}

\setlength{\oddsidemargin}{0mm}
\setlength{\textwidth}{150mm}

%\pagestyle{empty}

\maketitle

\section*{Document Control}

\begin{tabular}{|l|r|*{2}{p{.3\textwidth}|}}
\hline
\multicolumn{4}{|l|}{\texttt{'OETCS\_VV\_Process\_Synchronization\_TelCo\_Minutes\_130326.tex'}}
\\\hline
\textbf{Version} & \textbf{Date} & \textbf{Author} & \textbf{Changes/Comment}
\\\hline
01 & 2013-03-27 & Marc Behrens & All sections  
\\\hline
02 & 2013-03-28 & Hardi Hungar & Slight revision
\\\hline
\end{tabular}

\section*{Organizational Data}

\begin{tabular}{|l|r|r|}
\hline
\textbf{Type of meeting} & \multicolumn{2}{|c|}{ WebConf}
\\\hline
\textbf{Start} & 2013-03-26 & 11:00
\\\hline
 \textbf{End} & 2013-03-26 & 12:45
\\\hline
\end{tabular}

\medskip\noindent%

\begin{tabular}{|l|r|}
  \hline
\textbf{Participant} & \textbf{Organisation}
\\\hline
%Armand Nachtef & CEA List \\
Baseliyos Jacob& DB \\
%Cyril Cornu &All4tec \\
%Fr\'{e}d\'{e}rique Val\'{e}e & All4tec \\
Hardi Hungar & DLR \\
Jan Welte & TU-BS \\
%Jens Gerlach & Fraunhofer \\
Jo\~{a}o Santos & Institut Telecom \\
Klaus-R\"udiger Hase & DB \\
Marc Behrens & DLR \\
Marielle Petit-Doche & Systerel \\
%Martin Schr\"{o}der & ERA \\
Merlin Pokam & AEbt 
%Pierre-Fran\c{c}ois Jauquet & ALSTOM\\
%Piero Petroccioli & UIC \\
%Stephan Jagusch & AEbt \\
%Sylvain Baro & SNCF
\\\hline
\end{tabular}

\pagebreak

\renewcommand{\contentsname}{Agenda}
\label{sec:agenda}
\tableofcontents


\section*{Results}

% Use several tabular environments to split long result lists over pages

%\setcounter{section}{0}
%\setcounter{subsection}{0}
%use sections to document the agenda
\section{V\&V Process Synchronization} %1
\subsection{Synchronization on Agenda} %1.1
\setlength{\extrarowheight}{1.5pt}
\begin{longtable}{|p{0.83\textwidth}|p{.02\textwidth}|p{.15\textwidth}|}
% Description (free text)
% A (action item) OR D (decision) OR F (fact/finding) 
% responsible (for action items)
% 	deadline (for action items)
%
% in case of multi-page tables use the following sequence to end one page:
%		& T & Author
%		\end{longtable}
%        		\vskip 1 cm 
%		        \clearpage
%	    \begin{longtable}{|p{0.83\textwidth}|p{.02\textwidth}|p{.15\textwidth}|}
%	    \vskip 1 cm 
% header ------------------------
\hline
\textbf{Description} & \textbf{T} & \textbf{Resp.} 
%\hline
\endhead
\hline


%\Q{B.Jacob}{Who is responsible for safety properties?}

\textbf{Decision on safety functions:} The ones who start the model are responsible to identify the safety functions.

\C{M.Petit-Doche}{Safety Analysis and preliminary Hazard analysis to be done.}
\Q{B.Jacob}{Who is in charge of the safety analysis?}
\Q{B.Jacob}{How big will the scope be of the safety issues?}

\Q{K.-R.Hase}{What is the scope of the SSRS?}
\A{M.Petit-Doche}{SSRS is the functional architecture view.}

\Q{M.Petit-Doche}{Onboard only scope of the project?}
\A{K.-R.Hase}{Trackside has to be respected.}
\A{M.Behrens}{Agree there should be a trackside model defined, see Subset-026-2.4. Trackside model should be used for data preparation of the use case scenarios and test cases.}

\C{K.-R.Hase}{Clear interfaces are very important. The interface has to be described in the formalisation.}
\paragraph{Trackside Model}{It was agreed on to separate the SSRS into trackside (smaller data-preparation model) and on-board side (bigger part)}

& F/ \newline D
& Marc Behrens
\\\hline

\end{longtable}
%\vskip 1 cm 
%\clearpage

\subsection{V\&V Process Synchronization} %1.2

\begin{longtable}{|p{0.83\textwidth}|p{.02\textwidth}|p{.15\textwidth}|}
\hline
\textbf{Description} & \textbf{T} & \textbf{Resp.} 
%\hline
\endhead
\hline


	    \hline
%\subsection{V\&V Process Synchronization} %1.1.1
\begin{enumerate}
\item Each design artifact needs a reference artifact which it
  implements.  e.g.\ code to detailed model, detailed model to SRS
  model.
		
        Each step has to be verified. Saying this artifact comes from
        this part of the model: Verification needs a reference of each
        artifact of what should be implemented.
      \item The implementation relation shall be specified in detail,
        (e.g.\ for a state machine and a higher level state machine, a
        mapping of interfaces, states and transitions is required).
        This includes additional invariants, input assumptions and
        further restrictions. This information is the basis for
        verification activities.
		
        V\&V needs detailed references on parts of the model which are
        implemented.  Relation between these parts of the model is
        needed.  e.g.\ states of the concrete model map in a specific
        state mapping.

      \item The verifiability shall be incorporated within the model
        design. The same applies to the code. For the code, the
        standard (EN 50128) includes some explicit requirements for
        verifiability.

        Every designer has to keep verifiability in mind when
        performing some kind of implementation task.  At the very
        least, she/he should be able to justify the correctness of the
        implementation step (otherwise, the verifyer will most
        probably not be able to do his/her job). In order for the
        verifier to stand a chance on verifying explicit requirements
        for e.g.\ code verification should be anticipated beginning
        modelling.

      \item The findings from verification shall result in
        corrections.  Results can be:
		
        a) things we cannot verify
		
        b) the verification is able to identify detailed defects.

        Issues are reported back to the designer and need to be
        discussed and/or corrected.
		
        This feedback process from V\&V should be defined by WP4 and
        referenced in the QA-Plan.
	
        The design process should include its part of the feedback
        loop (clearing issues, correcting defects).
		
        \C{M.Petit-Doche}{Taking and checking the feedback from verification is a complex process.}

      \item Preliminary verification steps shall be performed and
        during model design and code development.


	Only stable code which passes basic functional tests, and only
        models which are reasonably consistent and complete and, if
        applicable, animated so that the main functions has been exercised
        should be the subject of a thorough verification. This is
        common practice as it is too costly to have a third party
        analyze an artifact which is most probably immature and buggy.
\end{enumerate}

		& D & Marc Behrens, \newline Hardi Hungar
		\end{longtable}
        		\vskip 1 cm 
		        \clearpage
	    \begin{longtable}{|p{0.83\textwidth}|p{.02\textwidth}|p{.15\textwidth}|}
	    %\vskip 1 cm 
	    \hline
	    \textbf{Description} & \textbf{T} & \textbf{Resp.} 
	    %\hline
	    \endhead
	    \hline
	    \newline

	
\Q{Agenda}{What are the requirements from V\&V influencing the other working streams and thus need to be predefined.}
\A{Meeting}{The process should be described by WP4}

\paragraph{Open questions}
\begin{enumerate}
	\item Which parts of the process do we expect to be described within the requirements?
	\item At what level does e.g. the V\&V plan fill in?
\end{enumerate}

& D
& Marc Behrens, \newline
Hardi Hungar
\\\hline
\end{longtable}

\bgcmmnt{\textbf{T} for type of item:
\begin{description}
	\item[A] action item
	\item[D] decision
	\item[F] fact / finding
\end{description}}



\section*{Notes}

This format lacks references to ITEA~2 so far.

% Optional for additional free text
  
\eod


 

\end{document}