%-----------------------------------------------------------------------
\section{Internal Assessment}
%-----------------------------------------------------------------------
One of the major point for a SIL4 compliant Software is the Whole Software Development Project Assessment by a Safety Authority (e.g CERTIFER in France, TÜV in Germany). As none of these companies are involved in openETCS Software Development assessment, the Internal Assessment objective is to simulate a real Assessor's tasks during the whole Open ETCS Software Development activities.

An assessment is a ¨ Process of analysis to determine whether software, which may include process, documentation, system, subsystem hardware and/or software components, meets the specified requirements, and to form a judgment whether the software is fit for its intended purpose. Safety assessment is focused on but not limited to the safety properties of a system.¨
%-----------------------------------------------------------------------
\subsection{Assessment tasks}
%-----------------------------------------------------------------------
The Assessor shall write a Software Assessment Plan. It is like an assessment process which is linked to the software development process.
More precisely, he shall explain the tasks needed to assess the software of the project OpenETCS.

{\itshape 
Note: The Verifier shall write a Software Assessment Verification Report, as required in the standard EN50128, to verify in the first time that the Software Assessment Plan meets the general requirements for readability and traceability.
}

During the software development, he shall evaluate the software verification and validation activities.
We propose that the Assessor intervenes at least seven times during the software development process (this is equivalent to one time at least by Work Product).

\textit{
Note: the numbers of work packages are not given in the chronological order, e.g. WP1 is performed during all the development process and WP5 occurs before the end of WP4.
}


\textbf{
During WP1: Project Management.
}

The Assessor is able to assess:
\begin{itemize}\itemsep=0pt
  \item The Quality Assurance
  \item The capability of the Project Manager and the quality of his deliverables
 \end{itemize}

The Assessor shall assess the Software Quality Plan. We propose that he gives a formal approval of this document.


\textbf{
During WP2: Requirements for Open Proof.
}


The Assessor is able to assess:
\begin{itemize}\itemsep=0pt
  \item The System requirements specification, including:
  \begin{itemize}\itemsep=3pt
    \item functions and interfaces;
    \item application conditions;
    \item configuration or architecture of the system;
    \item hazards to be controlled;
    \item safety integrity requirements;
    \item apportionment of requirements and allocation of SIL to software and hardware;
    \item timing constraints
   \end{itemize}
  \item The software requirements specification,
  \item The software architecture and design specification,
  \item The software component specification,
  \item The personnel key roles, responsibilities and competence,
  \item The Quality Assurance
 \end{itemize}
Nevertheless, he shall assess the implementation of both activities and deliverables of WP 2.


\textbf{
During WP3: Modeling of (part of) ETCS specification.
}

The Assessor shall evaluate the software implementation respectively the software modeling.
Furthermore, he is able to assess:
\begin{itemize}\itemsep=0pt
  \item A part of the life cycle and the documentation,
  \item The Quality Assurance,
  \item The personnel roles and responsibilities and competence.
 \end{itemize}
The Assessor shall assess the implementation of both activities and deliverables of WP 3.


\textbf{
During WP4: Validation \& Verification Strategy.
}

The Assessor shall assess:
\begin{itemize}\itemsep=0pt
  \item the Software Verification Plan and the Software Validation Plan,
  \item the Quality Assurance.
 \end{itemize}
We propose that he gives a formal approval on these documents. 
He shall mainly evaluate the verification activities and the implementation of both activities and deliverables of the WP 4.


\textbf{
During WP5: Demonstrator.
}

The Assessor shall assess the specific openETCS software.
Indeed, before the beginning of the validation activity (WP4), the Assessor shall assess the Software Integration Test Report to give or not the approval for software validation. This point is the Validation first step (the previous steps are related to the verification).


\textbf{
During WP6: Dissemination, Exploitation and Standardization.
}

The Assessor shall verify that the software maintenance plan is written and compliant with the software safety integrity level (SIL4).

\textbf{
During WP7: Language, Tool Chain and Open source Ecosystem.
}

The Assessor shall assess the developed tool chain according to Tool class T3 of EN 50128:2011. The other tools (T2) have to be assessed as well, but the effort is liter regarding the T3 assessment effort.

At the end of the software development process, the Assessor shall perform a final assessment. Indeed, he shall evaluate that the life cycle processes and products resulting are such that the software is of the defined software safety integrity level and fits for its intended application. All the steps of assessment performed during the software development process shall be gathered in the Software Assessment Report. This report could be updated all along the process.

{\itshape
Note: the Software Assessment Verification Report permit to verify the internal consistency of the Software Assessment Report.
}


\begin{table}[h]
\caption{T4.5 Inputs, Outputs and Deliverables} %title of the table
\begin{adjustbox}{width=\textwidth}
\begin{tabular}{|l|l|r|r|r|}
\hline
\multicolumn{5}{|c|}{\textbf{T4.5 Internal Assessment}} 
\\\hline
Type & Description & Due Date & Due Month & status 
%status output going to other tasks/wps    : not started, started, complete
%status input coming from other tasks/wps: no, yes
%\\\hline
%$\rightarrow$ & \todo{Ox.2.3: Sample Input Information}  & \shortmonthname[1]-2014 & T0+19 & no 
%\\\hline
%$\leftarrow$ & \todo{O4.5.1: Sample Output Information}   & \shortmonthname[10]-2013  & T0+16 & started  
\\\hline
 D &\emph{D 4.5} Quality recommendation to prepare the Assessment  & 29/03/2013 & March & 
\\\hline
\end{tabular}
\end{adjustbox}
\end{table}




