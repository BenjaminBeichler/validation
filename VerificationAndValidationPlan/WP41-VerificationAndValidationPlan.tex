\documentclass{template/openetcs_report}
% Use the option "nocc" if the document is not licensed under Creative Commons
%\documentclass[nocc]{template/openetcs_article} 
\usepackage{lipsum,url}
\usepackage{xspace}
\usepackage{graphicx}
\usepackage{fixme}
\usepackage{lscape} 
\usepackage{pgfgantt}
\usepackage{adjustbox}
\usepackage{datetime}
\usepackage{appendix}
\usepackage{enumerate}


%user specified macros
\newenvironment{activity}[2][planned]
	{\begin{tabular}{p{0.25\textwidth}@{\hspace{0.05\textwidth}}p{0.7\textwidth}}
			\multicolumn{2}{p{\textwidth}}{\colorbox{black}{\begin{minipage}{1.1cm}\begin{center}\textsc{\footnotesize \textcolor{white}{#1}}\end{center}\end{minipage}}~~\textbf{#2}}\\
	}
	{\end{tabular}}

\newcommand{\entry}[2]{#1:&#2\\}
\newcommand{\website}[1]{Website:&\url{#1}\\}
\newcommand{\desc}[1]{\multicolumn{2}{p{\textwidth}}{#1}\\}

\newcommand{\VV}{Verification \& Validation\xspace}
\newcommand{\vv}{verification \& validation\xspace}

\newcommand{\tbd}{\colorbox{cyan}{\%\%To Be Defined\%\%}}
\newcommand{\tbc}{\colorbox{cyan}{\%\%To Be Confirmed\%\%}}
\newcommand{\todo}[1]{\colorbox{cyan}{\%\%{#1}\%\%}}
\newcommand{\nthng}[1]{}
%% Requirements.


\newcounter{reqnum}
\setcounter{reqnum}{0}
\newcounter{subreqnum}
\newcounter{subsubreqnum}
\newlength{\partopbuf}
\newlength{\topbuf}

% Automated numbering versions of the macros
\newcommand{\req}[1]{\addtocounter{reqnum}{1} \setcounter{subreqnum}{0}
	\begin{description}\item[{\small\reqt-X-\thereqnum}] #1\end{description}
}

\newcommand{\subreq}[1]{
	\addtocounter{subreqnum}{1} \setcounter{subsubreqnum}{0}
	\addtolength{\leftmargini}{1cm}
	\begin{description}
	\item[\hspace{0.5cm}{\small\reqt-X-\thereqnum.\thesubreqnum}] #1
	\end{description}
	\addtolength{\leftmargini}{-1cm}
}

\newcommand{\subsubreq}[1]{
	\addtocounter{subsubreqnum}{1}
	\addtolength{\leftmargini}{2cm}
	\begin{description}
	\item[\hspace{1cm}{\small\reqt-X-\thereqnum.\thesubreqnum.\thesubsubreqnum}] #1
	\end{description}
	\addtolength{\leftmargini}{-2cm}
}

% Fixed version of the commands
\newcommand{\reqfixed}[3]{\addtocounter{reqnum}{1} \setcounter{subreqnum}{0}
	\begin{description}\item[{\small\reqt-#1-#2}] #3\end{description}
}

\newcommand{\subreqfixed}[4]{
	\addtocounter{subreqnum}{1} \setcounter{subsubreqnum}{0}
	\addtolength{\leftmargini}{1cm}
	\begin{description}
	\item[\hspace{0.5cm}{\small\reqt-#1-#2.#3}] #4
	\end{description}
	\addtolength{\leftmargini}{-1cm}	
}

\newcommand{\subsubreqfixed}[5]{
	\addtocounter{subsubreqnum}{1}
	\addtolength{\leftmargini}{2cm}
	\begin{description}
	\item[\hspace{1cm}{\small\reqt-#1-#2.#3.#4}] #5
	\end{description}
	\addtolength{\leftmargini}{-2cm}	
}

% Citation of the requirement

% Citation of the reference (for markup purpose)
%\newcommand{\refreq}[1]{\textbf{#1}}

% Citation of the reference and text (for markup purpose)
% The purpose of this is to automatically replace the placeholder by the 
% full text. \fullrefreq{R-xxx}{} or \fullrefreq{R-xxx}{blabla} 
% will be replaced by \fullrefreq{R-xxx}{text of the R-xxx requirement} 
%\newcommand{\fullrefreq}[2]{\textbf{#1}: \textrm{#2}}

\def\reqt{R-WP2/D2.6}
\newenvironment{justif}{
	\begin{quote}
	\begin{itshape}Justification. 
}{
	\end{itshape}
	\end{quote}
}

\newcommand{\cxx}{C\nolinebreak[4]\hspace{-.05em}\raisebox{.3ex}{\footnotesize\bf ++}\xspace}


\graphicspath{{./template/}{.}{./images/}}
\begin{document}
\frontmatter
\project{openETCS}

%Please do not change anything above this line
%============================
% The document metadata is defined below

%assign a report number here
\reportnum{OETCS/WP4/D4.1V00.02}

%define your workpackage here
\wp{Work Package 4: ``Validation \& Verification Strategy''}

%set a title here
\title{openETCS Validation \& Verification Plan}

%set a subtitle here
\subtitle{Version 00.02}

%set the date of the report here
\date{June 2013}

%define a list of authors and their affiliation here

\author{Marc Behrens and Hardi Hungar}

\affiliation{DLR\\
  Lilienthalplatz 7\\
  38108 Brunswick, Germany
   \\eMail:\{hardi.hungar,marc.behrens\}@dlr.de }

\author{Stephan Jagusch}

\affiliation{AEbt Angewandte Eisenbahntechnik GmbH\\
Adam-Klein-Str.\ 26\\
90429 N\"urnberg, Germany\\
eMail: Stephan.Jagusch@AEbt.de}
  
% define the coverart
\coverart[width=350pt]{openETCS_EUPL}

%define the type of report
\reporttype{Deliverable}



\begin{abstract}
%define an abstract here

  This document describes strategy and plan of the verification and
  validation activities in the project openETCS. As the goals of the
  project include the selection, adaption and construction of methods
  and tools for a FLOSS development in addition to performing actual
  development steps, differing from the plan for a full development
  project, the plan covers also activities evaluating the suitability
  of methods and tools, and it makes provisions for incorporation of
  V\&V of partial developments which are actually done.

  The overall strategy is to support the design process as specified
  in D2.3 and its partial instantiations within openETCS. In
  accordance with the project approach, V\&V shall be done in a FLOSS
  style, and it has to suit a model-based development. A further main
  consideration shall be to strive for conformance with the
  requirements of the standards (EN~50128 and further). This means
  that the contribution of all activities to a complete verification
  and validation shall be defined and assessed.

  The plan details how to perform \vv for a complete development which
  follows the process sketch from D2.3, so that the result conforms to
  the requirements of the standards for a SIL~4 development. This
  includes a definition of activities, the documentation to be
  produced, the organisation structure, roles, a selection of methods
  and tools, a format for describing design artifacts subject to V\&V,
  and a feedback format for the findings during V\&V.

  As D2.3 gives only a rough description of the development steps and
  not yet a complete list of design artifacts, nor one of methods
  applied and formats to be used, this first version of the V\&V plan
  will also lack detail which will to be added in later revisions as
  these informations become more concrete.

  Besides the usual purpose of \vv activities, namely evaluating and
  proving the suitability of design artifacts, V\&V in openETCS will
  also generate information on the suitability of the methods and tools
  employed. For that purpose, a format for describing methods
  and tools to be used in V\&V and one for summarizing the findings
  about the suitability are defined.

  The plan also contains partial instantiations of V\&V which match
  partial developments that are realised within openETCS.

\end{abstract}

%=============================
%Do not change the next three lines
\maketitle
\tableofcontents
\listoffiguresandtables
\newpage
%=============================

\begin{tabular}{|p{4.4cm}|p{8.7cm}|}
\hline
\multicolumn{2}{|c|}{Document information} \\
\hline
Work Package &  WP4  \\
Deliverable ID or doc.\ ref.\ & D4.1\\
\hline
Document title & openETCS Validation \& Verification Plan\\
Document version & 00.01 \\
Document authors (org.)  & Hardi Hungar (DLR), Marc Behrens (DLR),
Stephan Jagusch (AEbt) \\
\hline
\end{tabular}

\begin{tabular}{|p{4.4cm}|p{8.7cm}|}
\hline
\multicolumn{2}{|c|}{Review information} \\
\hline
Last version reviewed & -- \\
\hline
Main reviewers & -- \\
\hline
\end{tabular}

\begin{tabular}{|p{2.2cm}|p{4cm}|p{4cm}|p{2cm}|}
\hline
\multicolumn{4}{|c|}{Approbation} \\
\hline
  &  Name & Role & Date   \\
\hline  
Written by    &  Hardi Hungar & WP4-T4.1 Task Leader  &  June 2013\\
\hline
Approved by & -- & -- & \\
\hline
\end{tabular}

\begin{tabular}{|p{2.2cm}|p{2cm}|p{3cm}|p{5cm}|}
\hline
\multicolumn{4}{|c|}{Document evolution} \\
\hline
00.01 & 11/06/2013 & H. Hungar &  Document creation based on draft by
S. Jagusch\\
\hline
Version &  Date & Author(s) & Justification  \\
\hline  
00.02 & 14.06.2013 & J. Gerlach, H. Hungar &  Completed one
requirement table from draft, added detail to V\&V plan for full dev., \\
\hline  
\end{tabular}

% The actual document starts below this line
%=============================


%Start here

\chapter{Introduction}

\section{Purpose}
\label{sec:purpose}

The purpose of this document is to define the \vv activities in the
project openETCS.  

{\it This document describes strategy and plan of the
  verification and validation activities in the project openETCS. As
  the goals of the project include the selection, adaption and
  construction of methods and tools for a FLOSS development in
  addition to performing actual development steps, differing from the
  plan for a full development project, the plan covers also activities
  evaluating the suitability of methods and tools, and it makes
  provisions for incorporation of V\&V of partial developments which
  are actually done.}

\begin{description}
\item[WP4-T1-G:] A useful plan for WP 4, that is, one that defines a
way to achieve the goals of WP 4:
  \begin{description}
  \item[WP4-G1:] Identify and demonstrate methods and tools to handle
    the V\&V of a FLOSS development of the EVC software
  \item[WP4-G2:] Perform as much of V\&V on the DAS2Vs produced in the
    project as possible
  \end{description}
\end{description}

\paragraph{Detailed Goals and Means}
\label{sec:detailed-goals-means}

\begin{description}
\item[WP4-T1-G1:] The plan shall give an overview of and a structure to
  the things required from V\&V for an openETCS (FLOSS-) development.
  \begin{description}
  \item[WP4-T1-M1:] Identifies all (most) of the activities which have
    to be made for a full development according to the standards, in a
    form relevant to the approach of openETCS (FLOSS,
    participants). This may include alternatives.
  \end{description}
\item[WP4-T1-G2:] The plan shall provide a framework into which the V\&V
  activities which will be performed within the project do fit.
  \begin{description}
  \item[WP4-T1-M2-1:] Design formats for collecting information about
    DAS2Vs (V\&V tasks), about the results of V\&V activities, about
    activities of V\&V method and tool development, about the results
    of evaluations of V\&V methods and tools. Sketch how all of the
    information is to be gathered and finally incorporated into the
    final V\&V report (D4.4).
  \item[WP4-T1-M2-2:] Identify potential variants of partial
    implementations of V\&V processes which are likely going to be
    performed within the project. These may be (?should be?) related
    to design activities within the project which produce DAS2Vs.
  \end{description}
\item[WP4-T1-G3:] The plan shall delineate means for V\&V within openETCS
  \begin{description}
  \item[WP4-T1-M3-1:] A partial V\&V process (see WP4-T1-M2 above)
    consists of a set of related DASVs and V\&V steps to be applied to
    them. A V\&V step is described by input and output (result,
    purpose) with V\&V methods and means.
  \item[WP4-T1-M3-2:] The plan will prepare the selection of adequate
    methods and means (tools) by providing evaluation criteria and
    incorporating available evaluation results.
  \item[WP4-T1-M3-2-1:] Definition of an evaluation format for tools
    and methods.
  \end{description}
\item[WP4-T1-G4:] The plan shall incorporate currently available
  information on openETCS development process and means and be
  amendable to future changes and additions.
  \begin{description}
  \item[WP4-T1-M4-1:] Use D2.3 in instantiating the general
    requirements laid down in the standards.
  \item[WP4-T1-M4-2:] Use D2.1 for tools.
  \item[WP4-T1-M4-3:] Identify open points and include delineations
    for things which are useful for a complete V\&V but not yet
    planned or detailed by project activities already performed.
  \end{description}
\end{description}


This document describes which verification and validation activities
are needed for a full FLOSS development of the EVC software. It
describes how the work performed within the project openETCS is to be
organised to contribute to such a task, and how to demonstrate that it
can be realised.

The document is only valid in conjunction with the Quality Assurance
plan [1104G13-QA-plan]

\section{Plan for Completing this Document}
\label{sec:plan-completing-this}

{\it
\paragraph{Terminology}
\label{sec:terminology}
\begin{description}
\item [DAS2V:] Design Artifact Subject to Verification or Validation
\item[G:] Goal
\item[M:] Means
\item[F:] Finding/Result/Action
\end{description}
}


{\it 
\paragraph{Detailed Goals and Means}
\label{sec:detailed-goals-means}

\begin{description}
\item[WP4-T1-G1:] The plan shall give an overview of and a structure to
  the things required from V\&V for an openETCS (FLOSS-) development.
  \begin{description}
  \item[WP4-T1-M1:] Identifies all (most) of the activities which have
    to be made for a full development according to the standards, in a
    form relevant to the approach of openETCS (FLOSS,
    participants). This may include alternatives.
  \end{description}
\item[WP4-T1-G2:] The plan shall provide a framework into which the V\&V
  activities which will be performed within the project do fit.
  \begin{description}
  \item[WP4-T1-M2-1:] Design formats for collecting information about
    DAS2Vs (V\&V tasks), about the results of V\&V activities, about
    activities of V\&V method and tool development, about the results
    of evaluations of V\&V methods and tools. Sketch how all of the
    information is to be gathered and finally incorporated into the
    final V\&V report (D4.4).
  \item[WP4-T1-M2-2:] Identify potential variants of partial
    implementations of V\&V processes which are likely going to be
    performed within the project. These may be (?should be?) related
    to design activities within the project which produce DAS2Vs.
  \end{description}
\item[WP4-T1-G3:] The plan shall delineate means for V\&V within openETCS
  \begin{description}
  \item[WP4-T1-M3-1:] A partial V\&V process (see WP4-T1-M2 above)
    consists of a set of related DASVs and V\&V steps to be applied to
    them. A V\&V step is described by input and output (result,
    purpose) with V\&V methods and means.
  \item[WP4-T1-M3-2:] The plan will prepare the selection of adequate
    methods and means (tools) by providing evaluation criteria and
    incorporating available evaluation results.
  \item[WP4-T1-M3-2-1:] Definition of an evaluation format for tools
    and methods.
  \end{description}
\item[WP4-T1-G4:] The plan shall incorporate currently available
  information on openETCS development process and means and be
  amendable to future changes and additions.
  \begin{description}
  \item[WP4-T1-M4-1:] Use D2.3 in instantiating the general
    requirements laid down in the standards.
  \item[WP4-T1-M4-2:] Use D2.1 for tools.
  \item[WP4-T1-M4-3:] Identify open points and include delineations
    for things which are useful for a complete V\&V but not yet
    planned or detailed by project activities already performed.
  \end{description}
\end{description}
}

{\it
\paragraph{Concrete First Steps (in SCRUM terminology: the backlog)}
\label{sec:concrete-first-steps}

\begin{description}
\item[WP4-T1-S1:] Assess the input material
  \begin{description}
  \item[WP4-T1-S1-1:] Assess sketch of the V\&V plan (partly done)
    \begin{description}
    \item[WP4-T1-F1-1-1:] The current format is .doc
    \item[WP4-T1-F1-1-2:] The plan currently lists mainly the requirements
      on the plan and does not yet detail much of the plan itself.
    \item[WP4-T1-F1-1-3:] 
    \end{description}
  \item[WP4-T1-S1-2:] Assess D2.3 ``Process Definition'' with
    definition of DAS2Vs and V\&V steps
    \begin{description}
    \item[WP4-T1-F1-2-1:] DAS2Vs and \vv steps defined on a high level
    \end{description}
  \item[WP4-T1-S1-3:] Assess D2.9 ``Requirements for \VV''
    \begin{description}
    \item[WP4-T1-F1-3-1:] very high-level, requirements included in
      the appendix for reference in further completion in relevant for future steps 
    \end{description}
  \item[WP4-T1-S1-4:] Assess D2.1 (``Report on Existing Methodologies'')
    \begin{description}
    \item[WP4-T1-F1-4-1:] Seems very sketchy
    \end{description}
  \item[WP4-T1-S1-5:] Assess development and V\&V activities planned or
    already on the way for taking them into account in the V\&V plan 
    \begin{description}
    \item[WP4-T1-S1-5-1:] Ask a lot of people (or the right people)
    \item[WP4-T1-S1-5-1-1:] Design a query email (to be backed up by
      phone or personal inquiries) 
    \end{description}
  \item[WP4-T1-S2:] Organize the writing 
    \begin{description}
    \item[WP4-T1-S2-1:] Make a detailed work plan
    \item[WP4-T1-S2-1-1:] Transform the sketch to .tex
    \item[WP4-T1-S2-1-2:] Revise the structure according to what is
      expected to be done - accommodating the info on the process (D2.3
      -WP4-T1-S1-2) and on ongoing activities  (WP4-T1-S1-5). 
    \item[WP4-T1-S2-1-3:] References to the requirements (D2.9 - WP4-T1-S1-3) are to be included
    \item[WP4-T1-S2-1-4:] Tools and methods
    \item[WP4-T1-S2-1-4-1:] Format for evaluation (formulate evaluation criteria, D4.1a)
    \item[WP4-T1-S2-1-5:] Result collection
    \item[WP4-T1-S2-1-5-1:] Sketch all the formats (purpose)
    \item[WP4-T1-S2-1-5-2:] Sketch the process of information collection
      (T4.2 and T4.3 will have to do that) 
    \item[WP4-T1-S2-1-6:] Include section on V\&V plan revision
    \end{description}
  \item[WP4-T1-S2-2:] Find contributors
  \item[WP4-T1-S2-3:] Distribute the work
  \end{description}
\item[WP4-T1-S3:]  Do the work
\end{description}  
}

\section{Background Information}
\label{sec:backgr-inform}

\todo{Further Info, perhaps put the project context here }

\subsection{Definitions}

\paragraph{Verification}

Verification is an activity which has to be performed at each step of
the design. It has to be verified that the design step achieved its
goals. This consists at least of two parts:
\begin{itemize}
\item that the artifacts produced in the step are of the right type
  and contain allthe information they should. E.g., that the SSRS
  identifies all components addressed in SS~026, specifies their
  interfaces in sufficient detail and has allocated the functions to
  the components (this should just serve an example and is based on a
  guess what the SSRS should do)
\item that the artifact correctly implements the input requirements of
  the design step. These typically include the main output artifacts
  of the previous step. ``Correctly implements'' includes requirement
  coverage (tracing). This can and should be supported by some
  tools. Adequacy of such tools depends on things like format
  compatibility, degree of automation, functionality (e.g., ability to
  handle m-to-n relations). Depending on the design step (and the
  nature of the artifacts) different forms of verification will
  complement requirement coverage, with different levels of
  support. The step from SS~026 to the SSRS will mainly consist of
  manual activities besides things like coverage checks. Verifying a
  formal (executable) model against the SSRS can be supported by
  animation or simulation to e.g.\ execute test cases which have been
  designed to check compliance with the SSRS. Even formal proof tools
  may be employed to check or establish properties. Model-to-code steps
  offer far more options (and needs) for tool support. And tools or
  tool sets for unit test will support dynamic testing for requirement
  or code coverage. This may include test generation, test execution
  with report generation, test result evaluation and so on. Also, code
  generator verification (or qualification) may play a role,
  here. Integration steps mandate still other testing (or
  verification) techniques.
\end{itemize}
Summarizing, one may say that verification subsumes highly diverse
activities, and may be realized in very many different forms.

\paragraph{Validation}
%\nocite{*}
Validation is name for the activity by which the compliance of the end
result with the initial requirements is shown. In the case of
openETCS, this means that the demonstrator (or parts of it) are
checked against the SS~026 or one of its close descendants (i.e.,
SSRS). This will consist of testing the equipment according to a test
plan derived form the requirements and detailed into concrete test
cases at some later stage. Tool support for validation will thus
mainly concern test execution and evaluation, perhaps supplemented by
test derivation or test management. Ambitous techniques like formal
proof are most likely not applicable here.

Thus, the tool support for validation will not differ substantially
from that for similar verification activities.

One might also consider ``early'' validation activities, e.g.\
``validating'' an executable model against requirements from the
SS~026. These are not mandated by the standards and can per se not
replace design step verification. They may nevertheless be worthwhile
as means for early defect detection.

Further (mostly complementary) information on V\&V can be found in the
report on the CENELEC standards (D2.2).


\chapter{\VV Strategy}

{\it The overall strategy is to support the design process as
  specified in D2.3 and its partial instantiations within openETCS. In
  accordance with the project approach, V\&V shall be done in a FLOSS
  style, and it has to suit a model-based development. A further main
  consideration shall be to strive for conformance with the
  requirements of the standards (EN~50128 and further). This means
  that the contribution of all activities to a complete verification
  and validation shall be defined and assessed.  }

\section{\VV Strategy for a Full Development }
\label{sec:vv-strategy-full}

{\it 
Define the strategy for a full EVC software development.}

\section{\VV Strategy for openETCS}
\label{sec:vv-strategy-project}

{\it
The project will only perform part of the development, and thus also
only a part of the V\&V activities. Define how the project
demonstrates that a full development would be possible.
}

\chapter{\VV Plan for a Full Development}

\todo{detail}

{\it
Instantiate the generic \VV plan from the standard (and the draft) to
openETCS. That is, provide the requirements, define the design steps,
identify \vv activities to be performed and documents to be produced.}



{\it
 The plan details how to perform \vv for a complete development which
  follows the process sketch from D2.3, so that the result conforms to
  the requirements of the standards for a SIL~4 development. This
  includes a definition of activities, the documentation to be
  produced, the organisation structure, roles, a selection of methods
  and tools, a format for describing design artifacts subject to V\&V,
  and a feedback format for the findings during V\&V.

  As D2.3 gives only a rough description of the development steps and
  not yet a complete list of design artifacts, nor one of methods
  applied and formats to be used, this first version of the V\&V plan
  will also lack detail which will to be added in later revisions as
  these informations become more concrete.

  Besides the usual purpose of \vv activities, namely evaluating and
  proving the suitability of design artifacts, V\&V in openETCS will
  also generate information on the suitability of the methods and tools
  employed. For that purpose, a format for describing methods
  and tools to be used in V\&V and one for summarizing the findings
  about the suitability are defined.

  The plan also contains partial instantiations of V\&V which match
  partial developments that are realised within openETCS.}

\section{Plan Overview}
\label{sec:plan-overview}

\todo{A list of all steps, with input and output from Jagusch, adapted
to D2.3 steps}

\section{Requirements Base}
\label{sec:requirements-base}

The requirements on the EVC software origin in the SS-026 and TSI
specifications.

\todo{detail this, add references}

\section{Verification of the SSRS (Process Step 1c)}
\label{sec:verification-ssrs}

\subsection{Definition of the Object of Verification}
\label{sec:defin-SSRS}

The SSRS (sub-system requiement specification) outlines the subsystem
which is going to be modeled within the project. The SSRS describes
the architecture of the subsystem (functions and their I/O) and the
requirements allocated to these functions. If necessary, the
requirements are rewritten in order to address the I/O and to
correspond to the allocation. It also provides the classification into
vital and non vital requirements and data
streams. The architecture part is described in a semi-formal language,
and the requirements are described in natural language.

The SSRS is to be viewed as a supplement to the SS-026 and the
TSIs and is not intended to replace them. The verification has to
check that a complete and consistent set of functionalities have been
identified and that the architecture is adequate. Due to the informal
nature of the SSRS, mainly manual techniques are to be applied. 

\todo{Verifiy hazard analysis too?}

\subsection{Documentation to Be Created}
\label{sec:docum-SSRS}

\todo{define documents}

\section{Verification of the Subsystem Model (Process Step 2a)}
\label{sec:verif-subsyst-model}

\todo{add detail}



\section{Structure of the \VV Report}
\label{sec:structure-vv-plan}


{\it
The verification and validation plan covers the following central topics:
\begin{description}\setlength{\parsep}{0pt}\setlength{\itemsep}{0pt}\setlength{\topsep}{0pt}
%\reqfixed{04}{040}{x}
%\subreqfixed{04}{040}{1}{x}
\item[Header] containing all information to identify, this report, the
  authors, the approbation and reviewing entities.
\item[Executive Summary] giving an overview of the major elements from
  all sections. 
\item[Problem Statement] describing the challenges to be answered by
  \VV as well as the decisions to be taken based on the V\&V results
  as well as how to cope with potentially faulty output. It further
  describes the accreditation scope based on the risk assessment done
  on V\&V-level. 
\item[V\&V Requirements Traceability Matrix] links every V\&V artifact
  back to the requirements to measure e.g. test coverage and to
  directly link V\&V results to the requirements. 
\item[Acceptability Criteria,] describing the criteria for acceptance
  of the artifact into the \VV process e.g. as the direct translation
  of the requirements into metrics to measure success, are used
  e.g. for burndown charts within the process. 
\item[Assumptions] that are identified during the design of the
  verification and validation strategy and how these assumptions have
  an impact on the verdict by listing capabilities and limitations. 
\item[Risks and Impacts] that come across the execution of V\&V tasks
  together with the impacts foreseen. 
\item[V\&V Design] states how the V\&V process builds up including
  data preparation, execution and evaluation. 
\item[V\&V Methodologies] giving a step-by-step walkthrough of all
  possible V\&V activities including the assumptions, and
  verdict-relevant limitations and criteria for, e.g.,  model
  verification, model-to-code verification, unit testing, integration
  testing and final validation (according to the standard, this
  involves running the software on the target hardware).  
\item[V\&V Issues] describing unsolved V\&V issues and their impact on
  the affected proof or verdict. 
\item[Peer Reviews] going into details on how the community can take
  part and how official bodies and partners are integrated into the
  development and review process. 
\item[Test Plan Definition] going into the details of testing by
  describing among other things: 

\begin{description} 
\item[Title] as a unique identifier to the test plan.
\item[Description] of the test and the test-item giving information about version and revision.
\item[Features] to be tested and not to be tested in combination are
  listed together with information background.  
\item[Entry Criteria] which have to be met by the EVC before a test
  can be started, e.g. that the EVC has to be in level~3 limited
  supervision with the order to switch to level~2. 
\item[Suspension criteria and resumption requirements] are the central
  key to a smooth automation of the tests covering topics like
  \emph{when exiting this test before step 10, which entry criteria
    does it comply to or which resumption sequence has to be executed
    to continue testing}. 
\item[Walkthrough] covering a step-by-step approach of the test plan.
\item[Environmental requirements] going into the details of what is
  needed concerning the test environment, e.g. tools, adapter, data
  preparation. 
\end{description}

\item[Discrepancy Reports] identifying the defects.
\item[Key Participants] describing the assignment and task for each role involved. 

\begin{description}
\item[Accreditation of Participants] describing who was accredited to
  which role during the \VV phase. 
\item[V\&V Participants] listing the partners participating in V\&V activities,
\item[Other participants] including other interest groups such as
  reviewer by affiliate partners\footnote{affiliate partners are
    non-funded companies who signed the project cooperation agreement
    and with it get read access to the repositories starting from
    incubation phase to contribute e.g. by reviewing}. 
\end{description}

\item[Timeline] giving the timeline for the baselines as input to the
  V\&V process and identifying when each artifact should be created. 
\end{description}
}


\section{Methods and Tools}
\label{sec:methods-tools}

{\it The project shall select / develop / describe a chain of methods
  and tools for doing \vv in a full development.}


In common language, the notion {\em ``formal''} is often used in a
broad sense, meaning everything that can be described by rules, even
if they are rather vague.
%
Contrary to that, we use {\em ``formal''} in the narrow sense of
EN-50128 \cite[Section~D.28]{en50128},
meaning strictly mathematical techniques and methods.
%
Since the Aerospace Standard DO-178C \cite{DO-178C}
follows a similar understanding,
but gives more elaborate explanation in its supplementary document
%DO-333 
devoted to formal methods \cite{DO-333},
our presentation closely follows the terminology of the latter.


\begin{quote}
{\em Formal methods are mathematically based techniques for the
specification, development, and verification of software aspects of
digital systems.
%
The mathematical basis of formal methods consists
of formal logic, discrete mathematics, and computer-readable
languages.
%
The use of formal methods is motivated by the expectation
that, as in other engineering disciplines, performing appropriate
mathematical analyses can contribute to establishing the correctness
and robustness of a design.}

\hfill
\cite[Section~1.0, p.1]{DO-333}
\end{quote}


\subsection{Characterisation of Formal Methods}

Based on rigorous mathematical notions, formal methods may be used
to describe software systems' requirements in an unambiguous way,
thus supporting precise communication between engineers.
%
Formally specified requirements can be checked for consistency and
completeness by appropriate tools;
also, compliance between different representation levels of
specification can be verified.
%
Formal methods allow one to check software properties like:

\begin{itemize}
\item Freedom from exceptions
\item Freedom from deadlock
\item Non-interference between different levels of criticality
\item Worst case resource usage (execution time, stack, \ldots)
\item Correct synchronous or asynchronous behaviour,
        including absence of unintended behaviour
\end{itemize}


In order to subsume this variety of applications under a single
paradigm,
the DO-178C
considers a formal method to consist in applying a
formal {\em analysis} to a formal {\em model}.
%
Both analysis and model differs dependent on the particular method.
%
For most methods, the model
is just identical to the source code; however, it may
also be e.g.\ a tool-internally generated abstract state space (used
in the Abstract Interpretation method, cf.\
Section~\ref{sec:Abstract Interpretation} below).
%
For most methods, analysis tools need human advice;
however, they may also be fully automatic (e.g.\ for Model Checking, cf.\
\ref{sec:Model Checking}).

\subsection{Formal Analysis Methods}
\label{sec:formal-analysis}

In this section we present
the three most common methods for formal analysis.
The foundation of these analysis
methods are well understood and they have been
applied to many practical problems.


\subsubsection{Abstract Interpretation}
\label{sec:Abstract Interpretation}

The abstract interpretation method
\cite{Cousot.Cousot.1976}
builds at every point of a given program a conservative\footnote{
        i.e.\ guaranteeing soundness
}
representation
of the set of possible states that may occur there
during any execution run.
%
It determines particular effects of the program relevant for the
properties to be analysed, but does not actually execute it.
%
This allows one to statically determine dynamic properties of
infinite-state programs.
%
The main application is to check the absence of runtime errors, like
e.g.\ dereferencing of null-pointers, zero-divides,
and out-of-bound array accesses.
%
While conventional ad-hoc static analysis tools such as PCLint or \mbox{QA\cxx}
are well-tailored for quick, but incomplete analyses,
abstract-interpretation based tools require more computation time, but
guarantee that {\em all} runtime errors are detected, due to the
conservative
approximation of state sets.
%
Human intervention is required to improve the approximation accuracy
w.r.t.\ those program points where {\em false alarms}
have to be removed.


\subsubsection{Deductive Verification}

Deductive methods
\cite{Beckert.Marche.2010}
\cite{Ledinot.Pariente.2010}\nocite{Beckert.Marche.2010}
perform mathematical proofs to establish formally specified properties
of a given program, thus providing rigorous evidence.
%
Tools usually extract proof obligations from program code and property
specifications and attempt to prove them, automated or interactively.
%
Even automated prover tools usually need human assistance, e.g.\ by
providing loop invariants.


\subsubsection{Model Checking}
\label{sec:Model Checking}

Model checking
\cite{Clarke.Schlingloff.2001}\nocite{Robinson.Voronkov.2001}
explores all possible behaviours of a program to
determine whether a specified property is satisfied.
%
It is applicable only to programs with reasonable small state spaces;
the specifications are usually about temporal properties.
%
If a property is unsatisfied, a counter-example can be generated
automatically,
showing a use case leading to property violation.


\subsection{Verification with Formal Methods}

In the railway domain, the standard
EN~50128 highly recommends use of formal methods in
requirements specification (\cite[Table A.2]{en50128}),
software architecture (A.3),
software design and implementation (A.4),
verification and testing (A.5),
data preparation (A.11), and
modelling (A.17)
for Safety Integrity Level SIL~3 and above.
%
However, functional\slash black-box testing is still mandatory in
verification; this constraint may be considered as discouraging from
the use of formal methods.

Until recently, the situation was quite similar in the aerospace
domain.
%
J.\ Joyce, a member of the RTCA
standardisation committee SC-205, described
Airbus' problems in certifying their ``unit-proof for unit-test''
approach:

        \begin{quote}
        ``{\em Formal methods were used for certification credit in
        development of the A380, but apparently it was not a trivial
        matter to persuade certification authorities that this was
        acceptable even with the reference to formal methods in
        DO-178B as an alternative method.}''
        \end{quote}


Such experiences eventually caused the more detailed treatment of
formal method issues in the revision C of DO-178 that appeared in
late 2011.
%
The DO-178C considers formal methods as special cases of
reviews and analyses; thus incorporating them without major
structural changes of the software development recommendations.
%
For an employed formal method, the standard requires to justify its
unambiguity, its soundness\footnote{
        i.e., that the method never asserts a property to be true
        when it actually may be not true
},
and any additional assumptions\footnote{
        e.g.\ data range limits
}
needed by the method.
%
The DO-178C admits formal property verification on object code
as well as on source code, the latter additionally needing
evidence about property preservation of the source-to-object
code compiler.
%
However, ``{\em functional tests
executed in target hardware are always required to ensure that the
software in the target computer will satisfy the high-level requirements}''
\cite[FM.12.3.5]{DO-333}.

As a consequence of subsuming formal methods under general reviews and
analyses, no deviating special rules to qualify tools are necessary:
``{\em
Any tool that supports the formal analysis should be assessed under
the tool qualification
guidance required by DO-178C and qualified where necessary.}''
\cite[FM.1.6.2]{DO-333}.
Of course, for the railway domain, the rules of EN~50128 for supporting
software tools and languages must be taken into account
\cite[Section~6.7]{en50128}.


\section{Implementation of \VV}
\label{sec:implementation-vv}

The \vv has to be performed in cooperation with WP~3, which
produces DAS2Vs (models and code), and with WP~7, where methods
and tools are defined and developed. 

To exchange information with WP~3, formats are needed for collecting
information about DAS2Vs (V\&V tasks) and for giving back information
about the results of V\&V activities. Similarly, with WP~7
communication shall use formats to describe V\&V methods and tools
(input from WP~7) and the results of evaluations of V\&V methods and
tools.

\todo{Formats, activity organisation}


\chapter{\VV Plan for openETCS}

\todo{Describe how to proceed in openETCS to achieve the most.}

{\it
  \begin{itemize}
  \item \vv for partial developments
  \item evaluation
  \item demonstration story of capabilities
  \end{itemize}
}





\appendix
\chapter{Requirements on \VV}

\todo{Explain the requirement chapter.}
{\it
  \begin{itemize}
  \item Requirements from D2.9.
  \item Take the lists from the draft from 121207, retain the structure (at
    least preliminarily). 
  \end{itemize}
}

\section{Requirements on \VV from D2.9}
\label{sec:requirements-vv-D29}
\todo{Adapt the intro text} 

The already provided requirements require a
safety plan compliant to the CENELEC EN~50126, 50128 and 50129.  This
pulls a number of requirements on V\&V, including Verification and
Validation plans. On the topic of compliance to EN~50128, one shall
also refer to the D2.2 document.


\reqfixed{02}{061}{A Verification plan shall be issued and complied
  with.}  
\subreqfixed{02}{061}{01}{The verification plan shall
  provide a method to demonstrate the requirements covering all the
  development artifacts.}  
\subreqfixed{02}{061}{02}{The verification
  plan shall state all verification activities required for each of
  these development artifacts.} 
 \reqfixed{02}{062}{A Validation Plan
  shall be issued and complied with.}  
\subreqfixed{02}{062}{01}{The
  validation plan shall provide a method to validate all functional
  and safety requirements over all development artifacts.}
\subreqfixed{02}{062}{02}{The validation plan shall state all
  validation activities required for each of these development
  artifacts.}

\reqfixed{01}{021}{The test plan shall comply the mandatory documents
  of the SUBSET-076, restricted to the scope of the OpenETCS project.}
\begin{justif}
  It will possibly be difficult to model all the tests in the course
  of the project, but the test plan should at least be complete.
\end{justif}


\reqfixed{02}{063}{Each design artifact needs a reference artifact
  which it implements (\emph{e.g.} code to detailed model, SFM to SSRS
  model\dots)} 

\subreqfixed{02}{063}{01}{The implementation between them relation
  shall be specified in detail.}  e.g.\ for state machine and a higher
level state machine mapping of interfaces, states and transition is
required.  This includes additional invariants, input assumptions and
further restrictions. This informaiton is the basis for verification
activities.

\subreqfixed{02}{063}{02}{The design of the artifacts shall be made
  such to allow verifiability as far as possible.}

\reqfixed{02}{064}{The findings from the verification shall be traced,
  and will be adequately addressed (taken into consideration, or
  postponed or discarded with a justification).}



\section{General Requirements on Verification}

{\footnotesize\sffamily\centering
  \begin{longtable}{||p{.15\textwidth}|p{.4\textwidth}|p{.4\textwidth}||}
    \hline\hline
    \bfseries Excerpt from EN~50128:2011 [N01] & \bfseries
    Requirement & \bfseries Project Relevance\\
    \hline\hline
    \endhead
    \hline\hline
    \endfoot
    5.3.2.7 
    &
    For each document, traceability shall be provided in
    terms of a unique reference number and a defined and documented
    relationship with other documents. 
    &
    fully applicable\\
    \hline
    5.3.2.8 
    &
    Each term, acronym or abbreviation shall have the same meaning in every document.
    If, for historical  reasons, this is not possible,
    the different meanings shall be listed and the references given. 
    &
    \\
    \hline
    5.3.2.9 
    &
    Except for documents relating to pre-existing software (see 7.3.4.7),
    each document shall be written  according to the following rules:  
    \begin{itemize}
    \item it shall contain or implement all applicable conditions and
        requirements of the preceding document with which it has a hierarchical relationship;  
    \item it shall not contradict the preceding document.
    \end{itemize}
    &
    \\
    \hline
    5.3.2.10 
    &
    Each item or concept shall be referred to by the same name or description in every document. 
    &
    \\
    \hline
    6.5.4.14 
    &
    Traceability to requirements shall be an important consideration in the
    validation of a safety-related system and means shall be provided to allow
    this to be demonstrated throughout all phases of the lifecycle. 
    &
    \\
    \hline
    6.5.4.15 
    &
    Within the context of this European Standard,
    and to a degree appropriate to the specified software  safety integrity level,
    traceability shall particularly address  
    \begin{enumerate}[a)]
    \item  traceability of requirements to the design or other objects which fulfil them, 
    \item  traceability of design objects to the implementation objects which instantiate them.
    \item  traceability  of  requirements  and  design  objects  to  the  tests 
           (component,  integration,  overall  test)  and analyses that verify them.  
    \end{enumerate}

    Traceability shall be the subject of configuration management.  
    &
    \\
    \hline
    6.5.4.16
    &
    In special cases, e.g. pre-existing software or prototyped software,
    traceability may be established  after the implementation and/or
    documentation of the code, but prior to verification/validation.
    In these cases, it  shall  be  shown  that  verification/validation 
    is  as  effective  as  it  would  have  been  with  traceability  over  all phases. 
    &
    This requirement does’nt apply to the project.
    \\
    \hline
    6.5.4.17
    &
    Objects  of  requirements,  design  or  implementation  that  cannot  be 
    adequately  traced  shall  be demonstrated to have no bearing upon the safety
    or integrity of the system.
    &
    \\
    \hline
\end{longtable}}


{\footnotesize\sffamily\centering
  \begin{longtable}{||p{.15\textwidth}|p{.8\textwidth}||}
    \hline\hline
    \textbf{Excerpt from EN~50128:2011 [N01]} & \textbf{Requirement} \\
    \hline\hline
    \endhead
    \hline\hline
    \endfoot
    6.1.4.1
    &
    Tests performed by other parties such as the Requirements Manager, Designer or Implementer,
    if fully documented and complying with the following requirements, may be accepted by
    the Verifier.
    \\
    \hline
    6.1.4.2
    &
    Measurement equipment used for testing shall be calibrated appropriately.
    Any tools, hardware or software, used for testing shall be shown to be suitable for the purpose.
    \\
    \hline
    6.1.4.3
    &
    Software testing shall be documented by a Test Specification and a Test Report,
    as defined in the  following.
    \\
    \hline
    6.2.4.2
    &
    A Software Verification Plan shall be written,
    under the responsibility of the Verifier, on the basis of the necessary documentation.
    \\
    \hline
    6.2.4.3
    &
    The  Software  Verification  Plan  shall  describe  the  activities  to  be
    performed  to  ensure  proper  verification and that particular design or
    other verification needs are suitably provided for
    \\
    \hline
    6.2.4.4
    &
    During development (and depending upon the size of the system) the plan may be
    sub-divided into a number of child documents and be added to, 
    as the detailed needs of verification become clearer.
    \\
    \hline
    6.2.4.5
    &
    The Software Verification Plan shall document all the criteria, techniques and
    tools to be used in the  verification  process. 
    The  Software  Verification  Plan  shall  include  techniques  and  measures  chosen 
    from  Table A.5, Table A.6, Table A.7 and Table A.8.
    The selected combination shall be justified as a set satisfying  4.8, 4.9 and 4.10
    \\
    \hline
    6.2.4.6
    &
    The  Software  Verification  Plan  shall describe the activities to be
    performed to ensure correctness  and consistency with respect to the
    input to that phase. These include reviewing, testing and integration.
    \\
    \hline
    6.2.4.7
    &
    In  each  development  phase  it  shall  be  shown  that  the  functional, 
    performance  and  safety  requirements are met.
    \\
    \hline
    6.2.4.8
    &
    The results of each verification shall be retained in a format defined or
    referenced in the Software  Verification Plan.  
    \\
    \hline
    6.2.4.9
    &
    The Software Verification Plan shall address the following:  
    \begin{enumerate}[a)]
    \item  the selection of verification strategies and techniques
         (to avoid undue complexity in the assessment of the verification and testing,
         preference shall be given to the selection of techniques which are in
         themselves readily analysable);  
    \item  selection of techniques from Table A.5, Table A.6, Table A.7 and Table A.8;
    \item  the selection and documentation of verification activities;  
    \item  the evaluation of verification results gained;   
    \item  the evaluation of the safety and robustness requirements;  
    \item  the roles and responsibilities of the personnel involved in the verification process;  
    \item  the degree of the functional based test coverage required to be achieved;  
    \item  the structure and content of each verification step, especially for the
           Software Requirement Verification  (7.2.4.22), 
           Software  Architecture  and  Design  Verification  (7.3.4.41,  7.3.4.42), 
           Software  Components Verification (7.4.4.13),
           Software Source Code Verification (7.5.4.10) and
           Integration Verification (7.6.4.13) in a way that facilitates review against
           the Software Verification Plan.
    \end{enumerate}
    \\
    \hline
\end{longtable}}

\todo{Insert other tables.}

\nocite{*}
%===================================================
%Do NOT change anything below this line

\end{document}
