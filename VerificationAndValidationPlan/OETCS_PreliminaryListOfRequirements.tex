%Preliminary List of Requirements used for openETCS
\subsection{Preliminary List of Relevant Requirements}
"When designing a new on-board Control-Command and Signalling subsystem or when performing a major modification/upgrade of an existing subsystem where the application of the TSI \footnote{Technical Specification of Iteroperability/Control-Command and Signalling} is required[...]" \footnote{Guide for the application of the TSI for the Subsystem Control-Command and Signalling Trackside and On-board, ERA/GUI/07-2011/INT}

\subsubsection{Methodology of Extracting Requirements from Legal Reference}
The possible relevant requirements were identified starting out from the table \ref{tab:HighLevelRequirements} documents. 
The document [1] amended with document [2] represents the current decisions taken by the European Commission concerning 'High-speed rail system (HS) and conventional rail system (CR)' in its subsystem 'control-command and signalling'.
These documents will be referenced as TSI/CCS. 
The reference within the openETCS project will be the english version. 
From the total list of the mandatory requirements mentioned and referenced within the TSI/CCS the relevant requirements were identified  according to the scope 'functional On-Board Unit' of openETCS.
These requirements were then categorised to 
\begin{itemize}
\item central requirements documents, direclty referenced with on-board functionality
\item interface related requirements documents
\end{itemize}  
not taking into account that within the interface related requirements there may be functionality for central requirements   

\subsubsection{Requirements Formulated Within the TSI}
Still requirements explicitly formulated within the TSI and not being referenced within the TSI has to 

And taking the following hypothesis in account:
\begin{itemize} 
	\item ETCS Level 2 is not excluded.
	\item Class B Systems are not excluded.
\end{itemize}
The hypothesis shall be validated taking in account the openETCS operational scenarios.

\subsubsection{Disclaimer} 
The boundaries of the functional On-Board Unit within openETCS are not yet defined thus making this first draft document a not yet consolidated and non final or complete list of possible relevant requirements were extracted from the legal standards of table \ref{tab:HighLevelRequirements}.

\begin{table}[htbp]
	\centering
	\caption{High Level Reference documents}
	\setkeys{Gin}{keepaspectratio}
	\resizebox*{\textwidth}{\textheight}{
		% Table generated by Excel2LaTeX from sheet 'OETCS_HighLevelRequirements'
\begin{tabular}{rr|p{4.7cm}}
\toprule
Ref. No & Document Reference & Title \\
\midrule
$[1]$   & 
2012/88/EU & 
Commission Decision of 25 January 2012 on the technical specification for interoperability \
relating to the control-command and signalling subsystems of the trans-European rail system \\
$[2]$ &
2012/696/EU &
Commission Decision
of 6 November 2012
amending Decision 2012/88/EU on the technical specifications for interoperability relating to the control-command and signalling subsystems of the trans-European rail system\\
\bottomrule
\end{tabular}%

	}	
  \label{tab:HighLevelRequirements}%
\end{table}%


\subsubsection{List of Central Requirements}

 
\begin{table}[htbp]
	\centering
	\caption{Central Requirements}
	\setkeys{Gin}{keepaspectratio}
	\resizebox*{\textwidth}{\textheight}{
		% Table generated by Excel2LaTeX from sheet 'OETCS_Central_Requirements'
\begin{tabular}{rrrrrrr}
\toprule
%TSI Reference Number & Req. Reference & Req. Name & Req. Version & \begin{turn}{45}4.2.1.1. S a f e t y\end{turn} & \begin{turn}{45}4.2.2. On-board ERTMS/ETCS functionality\end{turn} & \begin{turn}{45}referring to\end{turn} \\
TSI Reference Number & 
Req. Reference & 
Req. Name & 
Req. Version & 
4.2.1.1. S a f e t y &
4.2.2. On-board ERTMS/ETCS functionality &
referring to \\
\midrule
27    & UNISIG SUBSET-091 & Safety Requirements for the Technical Interoperability of ETCS in Levels 1 and 2 & 3.2.0 & m     &       & safety \\
14    & UNISIG SUBSET-041 & Performance Requirements for Interoperability & 3.1.0 &       & m     & performance \\
4     & UNISIG SUBSET-026 & System Requirements Specification & 3.3.0 &       & m     & functions \\
13    & UNISIG SUBSET-040 & Dimensioning and
Engineering rules & 3.2.0 &       & m     & functions \\
60    & UNISIG SUBSET-104 & ETCS System Version Management & 3.1.0 &       & m     & functions \\
31    & Reserved UNISIG SUBSET-094 & Functional requirements for an on-board reference test facility &       &       & m     & tests \\
37 b  & Reserved UNISIG SUBSET-
076-5-2 & Test cases related to features &       &       & m     & tests \\
37 c  & Reserved UNISIG SUBSET-
076-6-3 & Test sequences &       &       & m     & tests \\
37 d  & Reserved UNISIG SUBSET-076-7 & Scope of the test specifications &       &       & m     & tests \\
23    & UNISIG SUBSET-054 & Responsibilities and rules for the assignment of values to ETCS variables & 3.0.0 &       & m     & ETCS-ID Management \\
\bottomrule
\end{tabular}%

	}	
  \label{tab:CentralRequirements}%
\end{table}%


\subsubsection{List of Interface Related Requirements}

 
\begin{table}[htbp]
	\centering
	\caption{Interface Related Requirements}
	\setkeys{Gin}{keepaspectratio}
	\resizebox*{\textwidth}{\textheight}{
		% Table generated by Excel2LaTeX from sheet 'OETCS_Interfaced_Requirements'
\begin{tabular}{
p{0.4cm}|
r
p{8.5cm}|
p{1.0cm}|
p{0.5cm}|
p{0.3cm}|
p{0.3cm}|
p{0.3cm}|
p{0.3cm}|
p{0.3cm}|
p{0.3cm}|
p{0.3cm}|
p{0.3cm}|
p{12.0cm}|
p{1.0cm}} %p{0.3cm}|
\toprule
\rotatebox{90}{TSI Reference Number} & 
Req. Reference & 
Req. Name & 
Req. Version & 
\rotatebox{90}{4.2.2.1. Communication with the Control-Command and Signalling Track-side Subsystem.}
& 
\rotatebox{90}{4.2.2.2 Communication with the driver}
&
\rotatebox{90}{4.2.2.3. Communicating with the STM}
&
\rotatebox{90}{4.2.2.4. Managing information about the completeness of the train}
& 
\rotatebox{90}{4.2.2.5. Equipment health monitoring and degraded mode support}
&
\rotatebox{90}{4.2.2.6. Support  data  recording for  regulatory purposes}
& 
\rotatebox{90}{4.2.2.7. Forwarding information/orders and receiving state information from rolling stock} 
& 
\rotatebox{90}{4.2.13. GSM-R DMI (Driver-Machine Interface)}
&
\rotatebox{90}{5.3 Constituents performance and specifications}
&
referring to &
optional if\\
\hline
\midrule
20    & UNISIG SUBSET-048 & Trainborne FFFIS
for Radio infill & 3.0.0 & m     &       &       &       &       &       &       &       &       & Radio communications with the train &  \\\hline
6     & ERA\_ERTMS\_
015560 & ETCS Driver
Machine interface & 3.3.0 &       & m     &       &       &       &       &       &       &       & communication with the driver &  \\\hline
7     & UNISIG SUBSET-034 & Train Interface FIS & 3.0.0 &       &       &       &       &       &       & m     &       &       & forwarding information/orders &  \\
\hline
64    & EN 301 515 & Global System for Mobile Communication (GSM); Requirements for GSM operation on railways & 2.3.0 & m     &       &       &       &       &       &       &       &       & Radio communications with the train: interface operating in GSM-R Band &  \\\hline
65    & TS 102 281 & Detailed requirements for GSM operation on railways & 2.2.0 & m     &       &       &       &       &       &       &       &       & Radio communications with the train: interface operating in GSM-R Band &  \\\hline
10    & UNISIG SUBSET-037 & EuroRadio  FIS & 3.0.0 & m     &       &       &       &       &       &       &       &       & Radio communications with the train: protocols &  \\\hline
39    & UNISIG SUBSET-092-1 & ERTMS EuroRadio Conformance Requirements & 3.0.0 & m     &       &       &       &       &       &       &       &       & Radio communications with the train: protocols &  \\\hline
40    & UNISIG SUBSET-092-2 & ERTMS EuroRadio test cases safety layer & 3.0.0 & m     &       &       &       &       &       &       &       &       & Radio communications with the train: protocols &  \\\hline
19    & UNISIG SUBSET-047 & Trackside- Trainborne FIS for Radio infill & 3.0.0 & o     &       &       &       &       &       &       &       &       & Radio communications with the train: radio in-fill & level 1 \\\hline
20    & UNISIG SUBSET-048 & Trainborne FFFIS
for Radio infill & 3.0.0 & o     &       &       &       &       &       &       &       &       & Radio communications with the train: radio in-fill & level 1 \\\hline
9     & UNISIG SUBSET-036 & FFFIS for
Eurobalise & 3.0.0 & m     &       &       &       &       &       &       &       &       & Eurobalise communication with the train &  \\\hline
43    & UNISIG SUBSET 085 & Test specification for Eurobalise FFFIS & 3.0.0 & m     &       &       &       &       &       &       &       &       & Eurobalise communication with the train &  \\\hline
16    & UNISIG SUBSET-044 & FFFIS for Euroloop & 2.4.0 & o     &       &       &       &       &       &       &       &       & Euroloop communication with the train & level 1 \\\hline
50    & UNISIG SUBSET-103 & Test specification for Euroloop & 1.1.0 & o     &       &       &       &       &       &       &       &       & Euroloop communication with the train & level 1 \\\hline
8     & UNISIG SUBSET-035 & Specific Transmission Module FFFIS & 3.0.0 &       &       & m     &       &       &       &       &       &       & transitions between ERTMS/ETCS and Class B train protection (if not using the standardised interface additional steps must be taken) &  \\\hline
25    & UNISIG SUBSET-056 & STM  FFFIS Safe time layer & 3.0.0 &       &       & m     &       &       &       &       &       &       & transitions between ERTMS/ETCS and Class B train protection (if not using the standardised interface additional steps must be taken) &  \\\hline
26    & UNISIG SUBSET-057 & STM  FFFIS Safe link layer & 3.0.0 &       &       & m     &       &       &       &       &       &       & transitions between ERTMS/ETCS and Class B train protection (if not using the standardised interface additional steps must be taken) &  \\\hline
36 c  & Reserved UNISIG SUBSET-074-2 & FFFIS STM  Test cases document &       &       &       & m     &       &       &       &       &       &       & transitions between ERTMS/ETCS and Class B train protection (if not using the standardised interface additional steps must be taken) &  \\\hline
49    & UNISIG SUBSET-059 & Performance requirements for STM & 3.0.0. &       &       & m     &       &       &       &       &       &       & transitions between ERTMS/ETCS and Class B train protection (if not using the standardised interface additional steps must be taken) &  \\\hline
52    & UNISIG SUBSET-058 & FFFIS STM  Application layer & 3.0.0 &       &       & m     &       &       &       &       &       &       & transitions between ERTMS/ETCS and Class B train protection (if not using the standardised interface additional steps must be taken) &  \\\hline
29    & UNISIG SUBSET-102 & Test specification for interface ?K? & 2.0.0 &       &       & o     &       &       &       &       &       &       & Interface K (to allow certain STMs to read information from Class B balises through the ERTMS/ETCS on-board antenna) & not implemented \\\hline
45    & UNISIG SUBSET-101 & Interface ?K? Specification & 2.0.0 &       &       & o     &       &       &       &       &       &       & Interface K (to allow certain STMs to read information from Class B balises through the ERTMS/ETCS on-board antenna) & not implemented \\\hline
46    & UNISIG SUBSET-100 & Interface ?G? Specification & 2.0.0 &       &       & m     &       &       &       &       &       &       & air gap between ETCS on- board antenna and Class B balises (if mandatory is depending on track project) &  \\\hline
34    & A11T6001 & (MORANE) Radio Transmission FFFIS for EuroRadio & 12.4  & m     &       &       &       &       &       &       &       &       & Interace between GSM-R Radio Data Communication and ERTMS/ETCS &  \\\hline
20    & UNISIG SUBSET-048 & Trainborne FFFIS
for Radio infill & 3.0.0 & o     &       &       &       &       &       &       &       &       & Interace between GSM-R Radio Data Communication and ERTMS/ETCS: radio in-fill & level 1 \\\hline
44    & Reserved & Odometry FIS &       &       &       &       &       &       &       &       &       & m     & Odometry &  \\
11    & UNISIG SUBSET-038 & Offline key management FIS & 3.0.0 & m     &       &       &       &       &       &       &       &       & Key Management &  \\\hline
6     & ERA\_ERTMS\_
015560 & ETCS Driver Machine interface & 3.3.0 &       & m     &       &       &       &       & m     &       &       & communication with the driver
forwarding information/orders &  \\\hline
80    & Reserved & GSM-R Driver Machine Interface &       &       &       &       &       &       &       &       & I     &       & GSM-R DMI (Driver-Machine Interface) &  \\\hline
5     & UNISIG SUBSET-027 & FIS Juridical Recording & 3.0.0 &       &       &       &       &       & m     &       &       &       & Data Recording for Regulatory Purposes &  \\\hline
      &       &       &       &       &       &       &       &       &       &       &       &       & mandatory for Level 3, no additional requirement document &  \\\hline
      &       &       &       &       &       &       &       &       &       &       &       &       & no additional requirement document
Functionalities:
- initialising the on-board ERTMS/ETCS functionality;
- providing degraded mode support;
- isolating the on-board ERTMS/ETCS functionality.
 &  \\\hline
      &       &       &       &       &       &       & m     &       &       &       &       &       &       & mandatory for Level 3, no additional requirement document \\\hline
\bottomrule
\end{tabular}%




	}	
  \label{tab:InterfaceRelatedRequirements}%
\end{table}%

