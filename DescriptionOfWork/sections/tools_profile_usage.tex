%-----------------------------------------------------------------------
\section{Identification of Tools and Profile Usage}
%-----------------------------------------------------------------------
The objective of this task is to prepare the activities of the tasks
T4.2 and T4.3, which are concerned with \vv of the model and the code,
respectively. It defines an overall \vv strategy and plans for
verification and validation, detailing how and with what means the
strategy is going to be implemented. Formally, the requirements on \vv
which are to be covered in the plans are listed in D2.9 ``Requirements
for \VV''. An essential requirement is adhering to
the applicable standards, mainly the EN~50128. The plans shall define
activities adequate for a complete development, but also foresee a
tailoring to the partial development actually realised in the project.

The WP~2 deliverable D2.3 ``Process Definition'' defines the
development process and its steps, and thus also identifies the main
verification steps. These will be detailed in the verification plan,
defining what has to be achieved on a more technical level. Also, a
selection of potentially applicable tools and methods will be
given. The validation plan shall take the overall development approach
including verification activities into account and define what methods
are suitable for demonstrating that all requirements (to be defined by
WP~2) are met by the end result, and also address the question of
safety integrity.

Thus, the plans shall contain a selection of methods and a list tools
suitable for applying the chosen V\&V methods for\footnote{Terms
  according to the draft of D2.3, abbreviations according to the draft
  of D2.6.}
\begin{enumerate}
\item the sub-system requirements specification (SSRS) and models
  (SFM, FFM)
\item the software semi-formal model and software architecture
  description
\item code derived from the software semi-formal model
\item the software strictly formal model and the software design
  description
\end{enumerate}

D2.1 ``Report on Existing Methodologies'' shall already provide a list
of potentially relevant methods and tools. Each method and tool
applied in WP~4 shall be described in a format detailing its purpose,
role and characteristics in terms of requirements on \vv.  Not all
steps will be automatic or semi-automatic. Manual techniques
like review or walkthrough will play a role, too. In selecting tools,
besides the requirement of openness (FLOSS), the question of
qualification, depending on the role the tool will play, has to be
answered. The formats describing methods and tools and criteria for
their evaluation will be given in a document D4.1a ``Preliminary
Evaluation Criteria on Verification and Validation'' supplementing the
deliverable to be produced by this task, D4.1 ``Report on V\&V Plan \&
Methodology''. 

Important classes of objects subjected to \vv activities are the
different models, be they semi-formal or formal, the code derived from
them and the versions of the demonstrator. Similar to methods and
tools, these objects need to be defined w.r.t.\ their nature and role
in the development process, including the requirements for \vv. A
format for their description shall be provided with D4.1.

An analysis of these objects and the methods with which they are
developed shall lead to a refinement and concretisation of the
verification and validation plan in the course of the project after
termination of T4.1. The plan is, as it is also common for any commercial
standard-conformant system development, a living object which has to
be revised and refined as the work proceeds. In the case of openETCS,
the tasks T4.2 and T4.3 will do that after the termination of
T4.1. The interim V\&V reports appear as suitable steps with which
this should happen. 


\begin{table}[h]
\caption{T4.1 Inputs, Outputs and Deliverables} %title of the table
\begin{adjustbox}{width=\textwidth}
\begin{tabular}{|l|l|r|r|r|}
\hline
\multicolumn{5}{|c|}{\textbf{T4.1 Identification of Tools and Profile Usage}} 
\\\hline
Type & Description & Due Date & Due Month & status 
%status output going to other tasks/wps    : not started, started, complete
%status input coming from other tasks/wps: no, yes
%\\\hline
%$\rightarrow$ & \todo{Ox.2.3: Sample Input Information}  & \shortmonthname[1]-2014 & T0+19 & no 
%\\\hline
%$\leftarrow$ & \todo{O4.1.1: Sample Output Information}   & \shortmonthname[10]-2013  & T0+16 & started  
\\\hline
$\rightarrow$ & \emph{D 2.1} Report on Existing Methodologies & \shortmonthname[3]-2013 & T0+9 &  no
\\\hline
$\rightarrow$ & \emph{D 2.3} Process Definition & \shortmonthname[5]-2013 & T0+11 &  no
\\\hline
$\rightarrow$ & \emph{D 2.4} Methods Definition & \shortmonthname[5]-2013 & T0+11 &  no
\\\hline
$\rightarrow$ & \emph{D 2.9} Set of Requirements for V\&V & \shortmonthname[5]-2013 & T0+11 &  no
\\\hline
 D &\emph{D 4.1a} Preliminary Evaluation Criteria on V\&V & \shortmonthname[3]-2013 & T0+9 & started
\\\hline
  D &\emph{D 4.1} Report on V\&V Plan \& Methodology  & \shortmonthname[7]-2013 & T0+13 & started
\\\hline
\end{tabular}
\end{adjustbox}
\end{table}

\subsection{Creating the Backlog}
\label{sec:creating-backlog}


\paragraph{Terminology}
\label{sec:terminology}
\begin{description}
\item [DAS2V:] Design Artifact Subject to Verification or Validation
\item[G:] Goal
\item[M:] Means
\item[F:] Finding/Result
\end{description}


\paragraph{Goal: Completion of the V\&V plan (in time)}
\label{sec:goal:-completion-vv}

\begin{description}
\item[WP4-T1-G:] A useful plan for WP 4, that is, one that defines a
way to achieve the goals of WP 4:
  \begin{description}
  \item[WP4-G1:] Identify and demonstrate methods and tools to handle
    the V\&V of a FLOSS development of the EVC software
  \item[WP4-G2:] Perform as much of V\&V on the DAS2Vs produced in the
    project as possible
  \end{description}
\end{description}

\paragraph{Detailed Goals and Means}
\label{sec:detailed-goals-means}

\begin{description}
\item[WP4-T1-G1:] The plan shall give an overview of and a structure to
  the things required from V\&V for an openETCS (FLOSS-) development.
  \begin{description}
  \item[WP4-T1-M1:] Identifies all (most) of the activities which have
    to be made for a full development according to the standards, in a
    form relevant to the approach of openETCS (FLOSS,
    participants). This may include alternatives.
  \end{description}
\item[WP4-T1-G2:] The plan shall provide a framework into which the V\&V
  activities which will be performed within the project do fit.
  \begin{description}
  \item[WP4-T1-M2-1:] Design formats for collecting information about
    DAS2Vs (V\&V tasks), about the results of V\&V activities, about
    activities of V\&V method and tool development, about the results
    of evaluations of V\&V methods and tools. Sketch how all of the
    information is to be gathered and finally incorporated into the
    final V\&V report (D4.4).
  \item[WP4-T1-M2-2:] Identify potential variants of partial
    implementations of V\&V processes which are likely going to be
    performed within the project. These may be (?should be?) related
    to design activities within the project which produce DAS2Vs.
  \end{description}
\item[WP4-T1-G3:] The plan shall delineate means for V\&V within openETCS
  \begin{description}
  \item[WP4-T1-M3-1:] A partial V\&V process (see WP4-T1-M2 above)
    consists of a set of related DASVs and V\&V steps to be applied to
    them. A V\&V step is described by input and output (result,
    purpose) with V\&V methods and means.
  \item[WP4-T1-M3-2:] The plan will prepare the selection of adequate
    methods and means (tools) by providing evualtion criteria and
    incorporating available evaluation results.
  \item[WP4-T1-M3-2-1:] Definition of an evaluation format for tools
    and methods.
  \end{description}
\item[WP4-T1-G4:] The plan shall incorporate currently available
  information on openETCS development process and means and be
  amendable to future changes and additions.
  \begin{description}
  \item[WP4-T1-M4-1:] Use D2.3 in instantiating the general
    requirements laid down in the standards.
  \item[WP4-T1-M4-2:] Use D2.1 for tools.
  \item[WP4-T1-M4-3:] Identify open points and include delineations
    for things which are useful for a complete V\&V but not yet
    planned or detailed by project activities already performed.
  \end{description}
\end{description}

\paragraph{Concrete First Steps (in SCRUM terminology: the backlog)}
\label{sec:concrete-first-steps}

\begin{description}
\item[WP4-T1-S1:] Assess the input material
  \begin{description}
  \item[WP4-T1-S1-1:] Assess sketch of the V\&V plan
    \begin{description}
    \item[WP4-T1-F1-1-1:] The current format is .doc
    \item[WP4-T1-F1-1-2:] The plan currently lists mainly the requirements
      on the plan and does not yet detail much of the plan itself.
    \end{description}
  \item[WP4-T1-S1-2:] Assess D2.3 ``Process Definition'' with
    definition of DAS2Vs and V\&V steps
    \begin{description}
    \item[WP4-T1-F1-2-1:] Very high-level
    \item[WP4-T1-F1-2-2:] Update expected?
    \end{description}
  \item[WP4-T1-S1-3:] Assess D2.9 ``Requirements for \VV''
    \begin{description}
    \item[WP4-T1-F1-3-1:] very very high-level, nothing except requirement
      numbers for reference relevant for future steps 
    \end{description}
  \item[WP4-T1-S1-4:] Assess D2.1 (``Report on Existing Methodologies'')
    \begin{description}
    \item[WP4-T1-F1-4-1:] Seems very sketchy
    \end{description}
  \item[WP4-T1-S1-5:] Assess development and V\&V activities planned or
    already on the way for taking them into account in the V\&V plan 
    \begin{description}
    \item[WP4-T1-S1-5-1:] Ask a lot of people (or the right people)
    \item[WP4-T1-S1-5-1-1:] Design a query email (to be backed up by
      phone or personal inquiries) 
    \end{description}
  \item[WP4-T1-S2:] Organize the writing 
    \begin{description}
    \item[WP4-T1-S2-1:] Make a detailed work plan
    \item[WP4-T1-S2-1-1:] Transform the sketch to .tex
    \item[WP4-T1-S2-1-2:] Revise the structure according to what is
      expected to be done - accommodating the info on the process (D2.3
      -WP4-T1-S1-2) and on ongoing activities  (WP4-T1-S1-5). 
    \item[WP4-T1-S2-1-3:] References to the requirements (D2.9 - WP4-T1-S1-3) are to be included
    \item[WP4-T1-S2-1-4:] Tools and methods
    \item[WP4-T1-S2-1-4-1:] Format for evaluation (formulate evaluation criteria, D4.1a)
    \item[WP4-T1-S2-1-5:] Result collection
    \item[WP4-T1-S2-1-5-1:] Sketch all the formats (purpose)
    \item[WP4-T1-S2-1-5-2:] Sketch the process of information collection
      (T4.2 and T4.3 will have to do that) 
    \item[WP4-T1-S2-1-6:] Include section on V\&V plan revision
    \end{description}
  \item[WP4-T1-S2-2:] Find contributors
  \item[WP4-T1-S2-3:] Distribute the work
  \end{description}
\item[WP4-T1-S3:]  Do the work
\end{description}  

