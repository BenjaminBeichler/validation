\documentclass{template/openetcs_report}
% Use the option "nocc" if the document is not licensed under Creative Commons
%\documentclass[nocc]{template/openetcs_article} 
\usepackage{lipsum,url}
\usepackage{xspace}
\usepackage{graphicx}
\usepackage{fixme}
\usepackage{lscape} 
\usepackage{pgfgantt}
\usepackage{adjustbox}
\usepackage{datetime}
\usepackage{appendix}



%user specified macros
\newenvironment{activity}[2][planned]
	{\begin{tabular}{p{0.25\textwidth}@{\hspace{0.05\textwidth}}p{0.7\textwidth}}
			\multicolumn{2}{p{\textwidth}}{\colorbox{black}{\begin{minipage}{1.1cm}\begin{center}\textsc{\footnotesize \textcolor{white}{#1}}\end{center}\end{minipage}}~~\textbf{#2}}\\
	}
	{\end{tabular}}

\newcommand{\entry}[2]{#1:&#2\\}
\newcommand{\website}[1]{Website:&\url{#1}\\}
\newcommand{\desc}[1]{\multicolumn{2}{p{\textwidth}}{#1}\\}

\newcommand{\VV}{Verification \& Validation\xspace}
\newcommand{\vv}{verification \& validation\xspace}

\newcommand{\tbd}{\colorbox{cyan}{\%\%To Be Defined\%\%}}
\newcommand{\tbc}{\colorbox{cyan}{\%\%To Be Confirmed\%\%}}
\newcommand{\todo}[1]{\colorbox{cyan}{\%\%{#1}\%\%}}
\newcommand{\nthng}[1]{}
%% Requirements.


\newcounter{reqnum}
\setcounter{reqnum}{0}
\newcounter{subreqnum}
\newcounter{subsubreqnum}
\newlength{\partopbuf}
\newlength{\topbuf}

% Automated numbering versions of the macros
\newcommand{\req}[1]{\addtocounter{reqnum}{1} \setcounter{subreqnum}{0}
	\begin{description}\item[{\small\reqt-X-\thereqnum}] #1\end{description}
}

\newcommand{\subreq}[1]{
	\addtocounter{subreqnum}{1} \setcounter{subsubreqnum}{0}
	\addtolength{\leftmargini}{1cm}
	\begin{description}
	\item[\hspace{0.5cm}{\small\reqt-X-\thereqnum.\thesubreqnum}] #1
	\end{description}
	\addtolength{\leftmargini}{-1cm}
}

\newcommand{\subsubreq}[1]{
	\addtocounter{subsubreqnum}{1}
	\addtolength{\leftmargini}{2cm}
	\begin{description}
	\item[\hspace{1cm}{\small\reqt-X-\thereqnum.\thesubreqnum.\thesubsubreqnum}] #1
	\end{description}
	\addtolength{\leftmargini}{-2cm}
}

% Fixed version of the commands
\newcommand{\reqfixed}[3]{\addtocounter{reqnum}{1} \setcounter{subreqnum}{0}
	\begin{description}\item[{\small\reqt-#1-#2}] #3\end{description}
}

\newcommand{\subreqfixed}[4]{
	\addtocounter{subreqnum}{1} \setcounter{subsubreqnum}{0}
	\addtolength{\leftmargini}{1cm}
	\begin{description}
	\item[\hspace{0.5cm}{\small\reqt-#1-#2.#3}] #4
	\end{description}
	\addtolength{\leftmargini}{-1cm}	
}

\newcommand{\subsubreqfixed}[5]{
	\addtocounter{subsubreqnum}{1}
	\addtolength{\leftmargini}{2cm}
	\begin{description}
	\item[\hspace{1cm}{\small\reqt-#1-#2.#3.#4}] #5
	\end{description}
	\addtolength{\leftmargini}{-2cm}	
}

% Citation of the requirement

% Citation of the reference (for markup purpose)
%\newcommand{\refreq}[1]{\textbf{#1}}

% Citation of the reference and text (for markup purpose)
% The purpose of this is to automatically replace the placeholder by the 
% full text. \fullrefreq{R-xxx}{} or \fullrefreq{R-xxx}{blabla} 
% will be replaced by \fullrefreq{R-xxx}{text of the R-xxx requirement} 
%\newcommand{\fullrefreq}[2]{\textbf{#1}: \textrm{#2}}

\def\reqt{R-WP2/D2.6}
\newenvironment{justif}{
	\begin{quote}
	\begin{itshape}Justification. 
}{
	\end{itshape}
	\end{quote}
}


\graphicspath{{./template/}{.}{./images/}}
\begin{document}
\frontmatter
\project{openETCS}

%Please do not change anything above this line
%============================
% The document metadata is defined below

%assign a report number here
\reportnum{OETCS/WP4/D4.1V01}

%define your workpackage here
\wp{Work Package 4: ``Validation \& Verification Strategy''}

%set a title here
\title{openETCS Validation \& Verification Plan}

%set a subtitle here
\subtitle{Version 01}

%set the date of the report here
\date{June 2013}

%define a list of authors and their affiliation here

\author{Marc Behrens and Hardi Hungar}

\affiliation{DLR\\
  Lilienthalplatz 7\\
  38108 Brunswick, Germany
   \\eMail:\{hardi.hungar,marc.behrens\}@dlr.de }

\author{Stephan Jagusch}

\affiliation{AEbt Angewandte Eisenbahntechnik GmbH\\
Adam-Klein-Str.\ 26\\
90429 N\"urnberg, Germany\\
eMail: Stephan.Jagusch@AEbt.de}
  
% define the coverart
\coverart[width=350pt]{openETCS_EUPL}

%define the type of report
\reporttype{Deliverable}



\begin{abstract}
%define an abstract here

  This document describes strategy and plan of the verification and
  validation activities in the project openETCS. As the goals of the
  project include the selection, adaption and construction of methods
  and tools for a FLOSS development in addition to performing actual
  development steps, differing from the plan for a full development
  project, the plan covers also activities evaluating the suitability
  of methods and tools, and it makes provisions for incorporation of
  V\&V of partial developments which are actually done.

  The overall strategy is to support the design process as specified
  in D2.3 and its partial instantiations within openETCS. In
  accordance with the project approach, V\&V shall be done in a FLOSS
  style, and it has to suit a model-based development. A further main
  consideration shall be to strive for conformance with the
  requirements of the standards (EN~50128 and further). This means
  that the contribution of all activities to a complete verification
  and validation shall be defined and assessed.

  The plan details how to perform \vv for a complete development which
  follows the process sketch from D2.3, so that the result conforms to
  the requirements of the standards for a SIL~4 development. This
  includes a definition of activities, the documentation to be
  produced, the organisation structure, roles, a selection of methods
  and tools, a format for describing design artifacts subject to V\&V,
  and a feedback format for the findings during V\&V.

  As D2.3 gives only a rough description of the development steps and
  not yet a complete list of design artifacts, nor one of methods
  applied and formats to be used, this first version of the V\&V plan
  will also lack detail which will to be added in later revisions as
  these informations become more concrete.

  Besides the usual purpose of \vv activities, namely evaluating and
  proving the suitability of design artifacts, V\&V in openETCS will
  also generate information on the suitability of the methods and tools
  employed. For that purpose, a format for describing methods
  and tools to be used in V\&V and one for summarizing the findings
  about the suitability are defined.

  The plan also contains partial instantiations of V\&V which match
  partial developments that are realised within openETCS.

\end{abstract}

%=============================
%Do not change the next three lines
\maketitle
\tableofcontents
\listoffiguresandtables
\newpage
%=============================

\begin{tabular}{|p{4.4cm}|p{8.7cm}|}
\hline
\multicolumn{2}{|c|}{Document information} \\
\hline
Work Package &  WP4  \\
Deliverable ID or doc. ref. & D4.1\\
\hline
Document title & openETCS Validation \& Verification Plan\\
Document version & 00.01 \\
Document authors (org.)  & Hardi Hungar (DLR), Marc Behrens (DLR),
Stephan Jagusch (AEbt) \\
\hline
\end{tabular}

\begin{tabular}{|p{4.4cm}|p{8.7cm}|}
\hline
\multicolumn{2}{|c|}{Review information} \\
\hline
Last version reviewed & -- \\
\hline
Main reviewers & -- \\
\hline
\end{tabular}

\begin{tabular}{|p{2.2cm}|p{4cm}|p{4cm}|p{2cm}|}
\hline
\multicolumn{4}{|c|}{Approbation} \\
\hline
  &  Name & Role & Date   \\
\hline  
Written by    &  Hardi Hungar & WP4-T4.1 Task Leader  &  June 2013\\
\hline
Approved by & -- & -- & \\
\hline
\end{tabular}

\begin{tabular}{|p{2.2cm}|p{2cm}|p{3cm}|p{5cm}|}
\hline
\multicolumn{4}{|c|}{Document evolution} \\
\hline
00.01 & 11/06/2013 & H. Hungar &  Document creation based on draft by
S. Jagusch\\
\hline
Version &  Date & Author(s) & Justification  \\
\hline  
-- & -- & -- &  --  \\
\hline  
\end{tabular}

% The actual document starts below this line
%=============================


%Start here

\chapter{Introduction}

\section{Purpose}
\label{sec:purpose}

The purpose of this document is to define the \vv activities in the
project openETCS.  

{\it This document describes strategy and plan of the
  verification and validation activities in the project openETCS. As
  the goals of the project include the selection, adaption and
  construction of methods and tools for a FLOSS development in
  addition to performing actual development steps, differing from the
  plan for a full development project, the plan covers also activities
  evaluating the suitability of methods and tools, and it makes
  provisions for incorporation of V\&V of partial developments which
  are actually done.}

\begin{description}
\item[WP4-T1-G:] A useful plan for WP 4, that is, one that defines a
way to achieve the goals of WP 4:
  \begin{description}
  \item[WP4-G1:] Identify and demonstrate methods and tools to handle
    the V\&V of a FLOSS development of the EVC software
  \item[WP4-G2:] Perform as much of V\&V on the DAS2Vs produced in the
    project as possible
  \end{description}
\end{description}

\paragraph{Detailed Goals and Means}
\label{sec:detailed-goals-means}

\begin{description}
\item[WP4-T1-G1:] The plan shall give an overview of and a structure to
  the things required from V\&V for an openETCS (FLOSS-) development.
  \begin{description}
  \item[WP4-T1-M1:] Identifies all (most) of the activities which have
    to be made for a full development according to the standards, in a
    form relevant to the approach of openETCS (FLOSS,
    participants). This may include alternatives.
  \end{description}
\item[WP4-T1-G2:] The plan shall provide a framework into which the V\&V
  activities which will be performed within the project do fit.
  \begin{description}
  \item[WP4-T1-M2-1:] Design formats for collecting information about
    DAS2Vs (V\&V tasks), about the results of V\&V activities, about
    activities of V\&V method and tool development, about the results
    of evaluations of V\&V methods and tools. Sketch how all of the
    information is to be gathered and finally incorporated into the
    final V\&V report (D4.4).
  \item[WP4-T1-M2-2:] Identify potential variants of partial
    implementations of V\&V processes which are likely going to be
    performed within the project. These may be (?should be?) related
    to design activities within the project which produce DAS2Vs.
  \end{description}
\item[WP4-T1-G3:] The plan shall delineate means for V\&V within openETCS
  \begin{description}
  \item[WP4-T1-M3-1:] A partial V\&V process (see WP4-T1-M2 above)
    consists of a set of related DASVs and V\&V steps to be applied to
    them. A V\&V step is described by input and output (result,
    purpose) with V\&V methods and means.
  \item[WP4-T1-M3-2:] The plan will prepare the selection of adequate
    methods and means (tools) by providing evaluation criteria and
    incorporating available evaluation results.
  \item[WP4-T1-M3-2-1:] Definition of an evaluation format for tools
    and methods.
  \end{description}
\item[WP4-T1-G4:] The plan shall incorporate currently available
  information on openETCS development process and means and be
  amendable to future changes and additions.
  \begin{description}
  \item[WP4-T1-M4-1:] Use D2.3 in instantiating the general
    requirements laid down in the standards.
  \item[WP4-T1-M4-2:] Use D2.1 for tools.
  \item[WP4-T1-M4-3:] Identify open points and include delineations
    for things which are useful for a complete V\&V but not yet
    planned or detailed by project activities already performed.
  \end{description}
\end{description}


This document describes which verification and validation activities
are needed for a full FLOSS development of the EVC software. It
describes how the work performed within the project openETCS is to be
organised to contribute to such a task, and how to demonstrate that it
can be realised.

The document is only valid in conjunction with the Quality Assurance
plan [1104G13-QA-plan]

\section{Plan for Completing this Document}
\label{sec:plan-completing-this}

{\it
\paragraph{Terminology}
\label{sec:terminology}
\begin{description}
\item [DAS2V:] Design Artifact Subject to Verification or Validation
\item[G:] Goal
\item[M:] Means
\item[F:] Finding/Result/Action
\end{description}
}


{\it 
\paragraph{Detailed Goals and Means}
\label{sec:detailed-goals-means}

\begin{description}
\item[WP4-T1-G1:] The plan shall give an overview of and a structure to
  the things required from V\&V for an openETCS (FLOSS-) development.
  \begin{description}
  \item[WP4-T1-M1:] Identifies all (most) of the activities which have
    to be made for a full development according to the standards, in a
    form relevant to the approach of openETCS (FLOSS,
    participants). This may include alternatives.
  \end{description}
\item[WP4-T1-G2:] The plan shall provide a framework into which the V\&V
  activities which will be performed within the project do fit.
  \begin{description}
  \item[WP4-T1-M2-1:] Design formats for collecting information about
    DAS2Vs (V\&V tasks), about the results of V\&V activities, about
    activities of V\&V method and tool development, about the results
    of evaluations of V\&V methods and tools. Sketch how all of the
    information is to be gathered and finally incorporated into the
    final V\&V report (D4.4).
  \item[WP4-T1-M2-2:] Identify potential variants of partial
    implementations of V\&V processes which are likely going to be
    performed within the project. These may be (?should be?) related
    to design activities within the project which produce DAS2Vs.
  \end{description}
\item[WP4-T1-G3:] The plan shall delineate means for V\&V within openETCS
  \begin{description}
  \item[WP4-T1-M3-1:] A partial V\&V process (see WP4-T1-M2 above)
    consists of a set of related DASVs and V\&V steps to be applied to
    them. A V\&V step is described by input and output (result,
    purpose) with V\&V methods and means.
  \item[WP4-T1-M3-2:] The plan will prepare the selection of adequate
    methods and means (tools) by providing evaluation criteria and
    incorporating available evaluation results.
  \item[WP4-T1-M3-2-1:] Definition of an evaluation format for tools
    and methods.
  \end{description}
\item[WP4-T1-G4:] The plan shall incorporate currently available
  information on openETCS development process and means and be
  amendable to future changes and additions.
  \begin{description}
  \item[WP4-T1-M4-1:] Use D2.3 in instantiating the general
    requirements laid down in the standards.
  \item[WP4-T1-M4-2:] Use D2.1 for tools.
  \item[WP4-T1-M4-3:] Identify open points and include delineations
    for things which are useful for a complete V\&V but not yet
    planned or detailed by project activities already performed.
  \end{description}
\end{description}
}

{\it
\paragraph{Concrete First Steps (in SCRUM terminology: the backlog)}
\label{sec:concrete-first-steps}

\begin{description}
\item[WP4-T1-S1:] Assess the input material
  \begin{description}
  \item[WP4-T1-S1-1:] Assess sketch of the V\&V plan (partly done)
    \begin{description}
    \item[WP4-T1-F1-1-1:] The current format is .doc
    \item[WP4-T1-F1-1-2:] The plan currently lists mainly the requirements
      on the plan and does not yet detail much of the plan itself.
    \item[WP4-T1-F1-1-3:] 
    \end{description}
  \item[WP4-T1-S1-2:] Assess D2.3 ``Process Definition'' with
    definition of DAS2Vs and V\&V steps
    \begin{description}
    \item[WP4-T1-F1-2-1:] DAS2Vs and \vv steps defined on a high level
    \end{description}
  \item[WP4-T1-S1-3:] Assess D2.9 ``Requirements for \VV''
    \begin{description}
    \item[WP4-T1-F1-3-1:] very high-level, requirements included in
      the appendix for reference in further completion in relevant for future steps 
    \end{description}
  \item[WP4-T1-S1-4:] Assess D2.1 (``Report on Existing Methodologies'')
    \begin{description}
    \item[WP4-T1-F1-4-1:] Seems very sketchy
    \end{description}
  \item[WP4-T1-S1-5:] Assess development and V\&V activities planned or
    already on the way for taking them into account in the V\&V plan 
    \begin{description}
    \item[WP4-T1-S1-5-1:] Ask a lot of people (or the right people)
    \item[WP4-T1-S1-5-1-1:] Design a query email (to be backed up by
      phone or personal inquiries) 
    \end{description}
  \item[WP4-T1-S2:] Organize the writing 
    \begin{description}
    \item[WP4-T1-S2-1:] Make a detailed work plan
    \item[WP4-T1-S2-1-1:] Transform the sketch to .tex
    \item[WP4-T1-S2-1-2:] Revise the structure according to what is
      expected to be done - accommodating the info on the process (D2.3
      -WP4-T1-S1-2) and on ongoing activities  (WP4-T1-S1-5). 
    \item[WP4-T1-S2-1-3:] References to the requirements (D2.9 - WP4-T1-S1-3) are to be included
    \item[WP4-T1-S2-1-4:] Tools and methods
    \item[WP4-T1-S2-1-4-1:] Format for evaluation (formulate evaluation criteria, D4.1a)
    \item[WP4-T1-S2-1-5:] Result collection
    \item[WP4-T1-S2-1-5-1:] Sketch all the formats (purpose)
    \item[WP4-T1-S2-1-5-2:] Sketch the process of information collection
      (T4.2 and T4.3 will have to do that) 
    \item[WP4-T1-S2-1-6:] Include section on V\&V plan revision
    \end{description}
  \item[WP4-T1-S2-2:] Find contributors
  \item[WP4-T1-S2-3:] Distribute the work
  \end{description}
\item[WP4-T1-S3:]  Do the work
\end{description}  
}

\section{Background Information}
\label{sec:backgr-inform}

\todo{Further Info, perhaps put the project context here }

\subsection{Definitions}

\paragraph{Verification}

Verification is an activity which has to be performed at each step of
the design. It has to be verified that the design step achieved its
goals. This consists at least of two parts:
\begin{itemize}
\item that the artifacts produced in the step are of the right type
  and contain allthe information they should. E.g., that the SSRS
  identifies all components addressed in SS~026, specifies their
  interfaces in sufficient detail and has allocated the functions to
  the components (this should just serve an example and is based on a
  guess what the SSRS should do)
\item that the artifact correctly implements the input requirements of
  the design step. These typically include the main output artifacts
  of the previous step. ``Correctly implements'' includes requirement
  coverage (tracing). This can and should be supported by some
  tools. Adequacy of such tools depends on things like format
  compatibility, degree of automation, functionality (e.g., ability to
  handle m-to-n relations). Depending on the design step (and the
  nature of the artifacts) different forms of verification will
  complement requirement coverage, with different levels of
  support. The step from SS~026 to the SSRS will mainly consist of
  manual activities besides things like coverage checks. Verifying a
  formal (executable) model against the SSRS can be supported by
  animation or simulation to e.g.\ execute test cases which have been
  designed to check compliance with the SSRS. Even formal proof tools
  may be employed to check or establish properties. Model-to-code steps
  offer far more options (and needs) for tool support. And tools or
  tool sets for unit test will support dynamic testing for requirement
  or code coverage. This may include test generation, test execution
  with report generation, test result evaluation and so on. Also, code
  generator verification (or qualification) may play a role,
  here. Integration steps mandate still other testing (or
  verification) techniques.
\end{itemize}
Summarizing, one may say that verification subsumes highly diverse
activities, and may be realized in very many different forms.

\paragraph{Validation}
%\nocite{*}
Validation is name for the activity by which the compliance of the end
result with the initial requirements is shown. In the case of
openETCS, this means that the demonstrator (or parts of it) are
checked against the SS~026 or one of its close descendants (i.e.,
SSRS). This will consist of testing the equipment according to a test
plan derived form the requirements and detailed into concrete test
cases at some later stage. Tool support for validation will thus
mainly concern test execution and evaluation, perhaps supplemented by
test derivation or test management. Ambitous techniques like formal
proof are most likely not applicable here.

Thus, the tool support for validation will not differ substantially
from that for similar verification activities.

One might also consider ``early'' validation activities, e.g.\
``validating'' an executable model against requirements from the
SS~026. These are not mandated by the standards and can per se not
replace design step verification. They may nevertheless be worthwhile
as means for early defect detection.

Further (mostly complementary) information on V\&V can be found in the
report on the CENELEC standards (D2.2).


\chapter{\VV Strategy}

{\it The overall strategy is to support the design process as
  specified in D2.3 and its partial instantiations within openETCS. In
  accordance with the project approach, V\&V shall be done in a FLOSS
  style, and it has to suit a model-based development. A further main
  consideration shall be to strive for conformance with the
  requirements of the standards (EN~50128 and further). This means
  that the contribution of all activities to a complete verification
  and validation shall be defined and assessed.  }

\section{\VV Strategy for a Full Development }
\label{sec:vv-strategy-full}

{\it 
Define the strategy for a full EVC software development.}

\section{\VV Strategy for openETCS}
\label{sec:vv-strategy-project}

{\it
The project will only perform part of the development, and thus also
only a part of the V\&V activities. Define how the project
demonstrates that a full development would be possible.
}

\chapter{\VV Plan for a Full Development}

\todo{detail}

{\it
Instantiate the generic \VV plan from the standard (and the draft) to
openETCS. That is, provide the requirements, define the design steps,
identify \vv activities to be performed and documents to be produced.}



{\it
 The plan details how to perform \vv for a complete development which
  follows the process sketch from D2.3, so that the result conforms to
  the requirements of the standards for a SIL~4 development. This
  includes a definition of activities, the documentation to be
  produced, the organisation structure, roles, a selection of methods
  and tools, a format for describing design artifacts subject to V\&V,
  and a feedback format for the findings during V\&V.

  As D2.3 gives only a rough description of the development steps and
  not yet a complete list of design artifacts, nor one of methods
  applied and formats to be used, this first version of the V\&V plan
  will also lack detail which will to be added in later revisions as
  these informations become more concrete.

  Besides the usual purpose of \vv activities, namely evaluating and
  proving the suitability of design artifacts, V\&V in openETCS will
  also generate information on the suitability of the methods and tools
  employed. For that purpose, a format for describing methods
  and tools to be used in V\&V and one for summarizing the findings
  about the suitability are defined.

  The plan also contains partial instantiations of V\&V which match
  partial developments that are realised within openETCS.}

\section{Structure of the \VV Report}
\label{sec:structure-vv-plan}


{\it
The verification and validation plan covers the following central topics:
\begin{description}\setlength{\parsep}{0pt}\setlength{\itemsep}{0pt}\setlength{\topsep}{0pt}
%\reqfixed{04}{040}{x}
%\subreqfixed{04}{040}{1}{x}
\item[Header] containing all information to identify, this report, the
  authors, the approbation and reviewing entities.
\item[Executive Summary] giving an overview of the major elements from
  all sections. 
\item[Problem Statement] describing the challenges to be answered by
  \VV as well as the decisions to be taken based on the V\&V results
  as well as how to cope with potentially faulty output. It further
  describes the accreditation scope based on the risk assessment done
  on V\&V-level. 
\item[V\&V Requirements Traceability Matrix] links every V\&V artifact
  back to the requirements to measure e.g. test coverage and to
  directly link V\&V results to the requirements. 
\item[Acceptability Criteria,] describing the criteria for acceptance
  of the artifact into the \VV process e.g. as the direct translation
  of the requirements into metrics to measure success, are used
  e.g. for burndown charts within the process. 
\item[Assumptions] that are identified during the design of the
  verification and validation strategy and how these assumptions have
  an impact on the verdict by listing capabilities and limitations. 
\item[Risks and Impacts] that come across the execution of V\&V tasks
  together with the impacts foreseen. 
\item[V\&V Design] states how the V\&V process builds up including
  data preparation, execution and evaluation. 
\item[V\&V Methodologies] giving a step-by-step walkthrough of all
  possible V\&V activities including the assumptions, and
  verdict-relevant limitations and criteria for, e.g.,  model
  verification, model-to-code verification, unit testing, integration
  testing and final validation (according to the standard, this
  involves running the software on the target hardware).  
\item[V\&V Issues] describing unsolved V\&V issues and their impact on
  the affected proof or verdict. 
\item[Peer Reviews] going into details on how the community can take
  part and how official bodies and partners are integrated into the
  development and review process. 
\item[Test Plan Definition] going into the details of testing by
  describing among other things: 

\begin{description} 
\item[Title] as a unique identifier to the test plan.
\item[Description] of the test and the test-item giving information about version and revision.
\item[Features] to be tested and not to be tested in combination are
  listed together with information background.  
\item[Entry Criteria] which have to be met by the EVC before a test
  can be started, e.g. that the EVC has to be in level~3 limited
  supervision with the order to switch to level~2. 
\item[Suspension criteria and resumption requirements] are the central
  key to a smooth automation of the tests covering topics like
  \emph{when exiting this test before step 10, which entry criteria
    does it comply to or which resumption sequence has to be executed
    to continue testing}. 
\item[Walkthrough] covering a step-by-step approach of the test plan.
\item[Environmental requirements] going into the details of what is
  needed concerning the test environment, e.g. tools, adapter, data
  preparation. 
\end{description}

\item[Discrepancy Reports] identifying the defects.
\item[Key Participants] describing the assignment and task for each role involved. 

\begin{description}
\item[Accreditation of Participants] describing who was accredited to
  which role during the \VV phase. 
\item[V\&V Participants] listing the partners participating in V\&V activities,
\item[Other participants] including other interest groups such as
  reviewer by affiliate partners\footnote{affiliate partners are
    non-funded companies who signed the project cooperation agreement
    and with it get read access to the repositories starting from
    incubation phase to contribute e.g. by reviewing}. 
\end{description}

\item[Timeline] giving the timeline for the baselines as input to the
  V\&V process and identifying when each artifact should be created. 
\end{description}
}


\section{Methods and Tools}
\label{sec:methods-tools}

{\it The project shall select / develop / describe a chain of methods
  and tools for doing \vv in a full development.}

\section{Implementation of \VV}
\label{sec:implementation-vv}

The \vv has to be performed in cooperation with WP~3, which
produces DAS2Vs (models and code), and with WP~7, where methods
and tools are defined and developed. 

To exchange information with WP~3, formats are needed for collecting
information about DAS2Vs (V\&V tasks) and for giving back information
about the results of V\&V activities. Similarly, with WP~7
communication shall use formats to describe V\&V methods and tools
(input from WP~7) and the results of evaluations of V\&V methods and
tools.

\todo{Formats, activity organisation}


\chapter{\VV Plan for openETCS}

\todo{Describe how to proceed in openETCS to achieve the most.}

{\it
  \begin{itemize}
  \item \vv for partial developments
  \item evaluation
  \item demonstration story of capabilities
  \end{itemize}
}





\appendix
\chapter{Requirements on \VV}

\todo{Explain the requirement chapter.}
{\it
  \begin{itemize}
  \item Requirements from D2.9.
  \item Take the lists from the draft from 121207, retain the structure (at
    least preliminarily). 
  \end{itemize}
}

\section{Requirements on \VV from D2.9}
\label{sec:requirements-vv-D29}
\todo{Adapt the intro text} 

The already provided requirements require a
safety plan compliant to the CENELEC EN~50126, 50128 and 50129.  This
pulls a number of requirements on V\&V, including Verification and
Validation plans. On the topic of compliance to EN~50128, one shall
also refer to the D2.2 document.


\reqfixed{02}{061}{A Verification plan shall be issued and complied
  with.}  
\subreqfixed{02}{061}{01}{The verification plan shall
  provide a method to demonstrate the requirements covering all the
  development artifacts.}  
\subreqfixed{02}{061}{02}{The verification
  plan shall state all verification activities required for each of
  these development artifacts.} 
 \reqfixed{02}{062}{A Validation Plan
  shall be issued and complied with.}  
\subreqfixed{02}{062}{01}{The
  validation plan shall provide a method to validate all functional
  and safety requirements over all development artifacts.}
\subreqfixed{02}{062}{02}{The validation plan shall state all
  validation activities required for each of these development
  artifacts.}

\reqfixed{01}{021}{The test plan shall comply the mandatory documents
  of the SUBSET-076, restricted to the scope of the OpenETCS project.}
\begin{justif}
  It will possibly be difficult to model all the tests in the course
  of the project, but the test plan should at least be complete.
\end{justif}


\reqfixed{02}{063}{Each design artifact needs a reference artifact
  which it implements (\emph{e.g.} code to detailed model, SFM to SSRS
  model\dots)} 

\subreqfixed{02}{063}{01}{The implementation between them relation
  shall be specified in detail.}  e.g.\ for state machine and a higher
level state machine mapping of interfaces, states and transition is
required.  This includes additional invariants, input assumptions and
further restrictions. This informaiton is the basis for verification
activities.

\subreqfixed{02}{063}{02}{The design of the artifacts shall be made
  such to allow verifiability as far as possible.}

\reqfixed{02}{064}{The findings from the verification shall be traced,
  and will be adequately addressed (taken into consideration, or
  postponed or discarded with a justification).}



\section{General Requirements on Verification}

{\footnotesize\sffamily\centering
  \begin{longtable}{||p{.15\textwidth}|p{.4\textwidth}|p{.4\textwidth}||}
    \hline\hline
    \bfseries Excerpt from EN~50128:2011 [N01] & \bfseries
    Requirement & \bfseries Project Relevance\\
    \hline\hline
    \endhead
    \hline\hline
    \endfoot
    5.3.2.7 & For each document, traceability shall be provided in
    terms of a unique reference number and a defined and documented
    relationship with other documents. 
    & fully applicable\\
    \hline
\end{longtable}}

\todo{Complete the table.}

\todo{Insert other tables.}

\nocite{*}
%===================================================
%Do NOT change anything below this line

\end{document}
