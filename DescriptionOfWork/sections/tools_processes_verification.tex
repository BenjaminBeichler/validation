%-----------------------------------------------------------------------
\section{Verification of the Tools and Processes}
%-----------------------------------------------------------------------
Since one of the openETCS project goals is to define processes and a corresponding tools which are suited to develop ETCS train Onboard Unit software based on a model of the system requirement specifications, the development processes have to follow the CENELEC Standards foremost EN 50128. As this should be done using open source principles and agile development methods it has to be demonstrated that all processes have been applied properly respecting the required principles. Therefore all life cycle stages (SW Requirement Specification, SW Design, SW Coding, etc.) and their related verification and validation activities have to be designed, performed and documented consistently and as required by the CENELEC standards. 

The work of this task 4.4 will verify proper design, performance and documentation of the overall development process and its corresponding tools by inspection of the respective documents. According to EN 50129 a safety plan as documentary evidence that during the development of an safety-related electronic railway system the conditions for safety acceptance a satisfied has to be submitted to the relevant safety authorities. Therefore a safety case covers evidence of quality management, safety management, and functional and technical safety in a structured justification. Since  the main product of the openETCS project shall be a non-vital demonstrator implementation it is unnecessary and infeasible for task 4.4 to write a complete safety case. First and foremost the documentation of the actual openETCS development will cover quality management aspects (among others):
\begin{enumerate}
\item  Documentation of the development process
\item  Roles, responsibilities and competences of the involved bodies
\item  Traceability during the development
\item  Documentation control and Configuration management
\item  Fault management
\item  Grievance handling
\item  Modification and change control
\item  Tool functionality and tool handling documentation
\end{enumerate}

The safety related verification of tools and processes has to deal with the to aspects, safety management and the functional (and technical) safety. However, the openETCS project will not deliver a vital software implementation consequently most of the activities will not or only for demonstration purposes be performed in the project. Respectively instead of verifying and documenting the actual activities in a safety case, the work of task 4.4 will rather be to present a generic justification structure for a safety case, which can be used in following projects or by the industrial partners for the development of an openETCS based Onboard Unit. The partial execution of safety activities shall be used to demonstrated the structure and to show their serviceability. 

To reach this goal the safety management team working in task 4.4 will design a safety plan to outline the overall structure of the safety management and the pursued way to demonstrate the functional safety. Accordingly the sequence of safety activities will be identified including all verification and validation activities for safety requirements. Overall the safety plan will describe the following points:
\begin{enumerate}
\item  Safety specific roles, responsibilities and competences of the involved bodies
\item  Principles and overall process of the safety management
\item  Requirements on the different tools supporting the development process and criteria for the tool categorization
\item  Safety related activities and all their corresponding documents to prove functional (and technical) safety (including those for the supporting tools)
\item  Structure of the safety case based on the documentation of the safety activities and their supporting tools
\item  Principles and procedures for preparing the safety case
\item  Specific procedures for maintaining the functional safety and the corresponding safety documents over time
\end{enumerate}

All these activities are closely connected to the design, verification and validation activities and therefore have to be defined considering the input of this teams. Especially concerning the applied tools, it is likely that 3rd parties have to be engaged to perform the verification and validation or deliver the needed documentation.

Following the safety plan sample activities shall be chosen to be performed on part of the Onboard unit. This shall give evidence that the presented safety management is exercisable and can be serviced by the used tools. Additionally it shall demonstrate the safeguards used for individual safety properties to ensure functional and technical safety. 
Since in a model based development the hazard identification, risk assessment and safety requirement verification is very closely related to the design activities and the supporting tools, safety activities have to be performed mainly by other teams in the development process. The safety team of task 4.4 itself can only guide these activities and process the results to present them in the safety case. 




%Specify the tool classes and the potentials to Categorize which tools have to be T3 and which T2.}

%Safety Evaluation criteria

\begin{table}[h]
\caption{T4.4 Inputs, Outputs and Deliverables} %title of the table
\begin{adjustbox}{width=\textwidth}
\begin{tabular}{|l|l|r|r|r|}
\hline
\multicolumn{5}{|c|}{\textbf{T4.4 Verification of the Tools and Processes}} 
\\\hline
Type & Description & Due Date & Due Month & status 
%status output going to other tasks/wps    : not started, started, complete
%status input coming from other tasks/wps: no, yes

%\\\hline
%$\rightarrow$ & \todo{Ox.2.3: Sample Input Information}  & \shortmonthname[1]-2014 & T0+19 & no 
%\\\hline
%$\leftarrow$ & \todo{O4.2.1: Sample Output Information}   & \shortmonthname[10]-2013  & T0+16 & started 
\\\hline
$\rightarrow$ & \emph{D 1.3.1} Project Guide on Quality Assurance  & \shortmonthname[6]-2013 & T0+12 & no 
\\\hline
$\rightarrow$ &	\emph{D 2.4} Methods definition & \shortmonthname[2]-2013 & ? & no
\\\hline
$\rightarrow$ &	\emph{D 2.6-9} Set of Requirements & \shortmonthname[6]-2013 & ? & no
\\\hline

$\rightarrow$ &	\emph{D 7.1} Report on the final choice(s) for the primary tool chain (means of description, tool and platform) & \shortmonthname[6]-2013 & ? & no
\\\hline
$\rightarrow$ &	\emph{D 7.2} Report on all aspects of secondary tooling & \shortmonthname[6]-2013 & ? & no
\\\hline
$\rightarrow$ &	\emph{I: O 7.2.8} Safety analyses tools choices & \shortmonthname[6]-2013 & ? & no
\\\hline
$\rightarrow$ &	\emph{D 7.3.1.2} Tools Interoperability Description & \shortmonthname[6]-2013 & ? & no
\\\hline
$\rightarrow$ &	\emph{I: O 7.3.1} Tool chain development plan (or equivalent) & \shortmonthname[6]-2013 & ? & no
\\\hline
$\rightarrow$ &	\emph{I: O 7.3.2} Specification of tool interoperability mechanisms & \shortmonthname[6]-2013 & ? & no
\\\hline
$\rightarrow$ &	\emph{I: O 7.3.4} Specification of primary and support tool chain architecture and its embedding into the platform & \shortmonthname[6]-2013 & ? & no
\\\hline
$\rightarrow$ &	\emph{D 7.3} Tool Chain Qualification Process Description & \shortmonthname[6]-2013 & ? & no
\\\hline
$\rightarrow$ &	\emph{D 7.4} Tool chain first release & \shortmonthname[2]-2014 & T0+20 & no
\\\hline
$\rightarrow$ & \emph{WP3} Feedback concerning the quality and safety management processes	 & \shortmonthname[6]-2013 & ? & no
\\\hline
$\rightarrow$ & \emph{WP3} Feedback concerning potential hazards & \shortmonthname[6]-2013 & ? & no
\\\hline
$\rightarrow$ & \emph{D 4.1} Report on V\&V Plan \& Methodology & \shortmonthname[7]-2013 & T0+13 & no
\\\hline
$\rightarrow$ & \todo{I: O Task 4.2}  & \shortmonthname[7]-2013 & ? & no
\\\hline
$\rightarrow$ & \todo{I: O Task 4.3}  & \shortmonthname[7]-2013 & ? & no
\\\hline
$\rightarrow$ & \emph{WP4 - T 4.5} Feedback concerning documentation quality  & \shortmonthname[7]-2013 & ? & no

\\\hline
$\leftarrow$ & \emph{O 4.4.1} Safety Plan   & \shortmonthname[10]-2013  & T0+16 & started 
\\\hline
$\leftarrow$ & \emph{O 4.4.2} Report on safety demonstration activities   & \shortmonthname[2]-2014  & T0+20 & not started 

\\\hline
D & \emph{D 4.4} Final report concerning the Safety Case  & \shortmonthname[6]-2015 & T0+36 & not started
\\\hline
\end{tabular}
\end{adjustbox}
\end{table}